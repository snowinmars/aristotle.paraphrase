\documentclass{article}

\usepackage[T2A]{fontenc}
\usepackage[utf8]{inputenc}
\usepackage[russian]{babel}

\linespread{1.1}
\setlength{\parskip}{1em}
\usepackage[left=1cm,right=1cm,top=1cm,bottom=2cm]{geometry}

\begin{document}

Если мы хотим овладеть наукой мудрости, то начать следует с поиска её начал. Каков стереотип мудрого человека? Тот мудр, кто: <ul><li>знает всё, хотя не знает каждый предмет в отдельности,</li><li>способен познать трудное для обычного человека. Отмечу отдельно, что восприятие свойственно всем, поэтому в нём ничего мудрого нет,
\footnotemark[1]
</li><li>точен и способен научить самой мудрости - выявлению причин.</li></ul>

Можно указать и свойства самой мудрости как науки. Из всех наук мудрость больше та, которая <ul><li>существует ради себя и ради познания, нежели та, которая существует ради извлекаемой из неё выгоды,</li><li>главенствует, нежели обслуживает. Ибо мудрости надлежит наставлять, а не повиноваться.</li></ul>

Следовательно, тот мудр, кто обладает знанием общего. Ведь в некотором смысле он знает всё, что попадает под "общее". Но "общее" познавать труднее всего: оно расположено очень далеко от непосредственного, чувственного восприятия.

Точными называются такие науки, которые явно выделяют свои начала - аксиомы. И чем точнее и строже наука, тем меньше в ней начал. Например, арифметика строже химии. Наука тем эффективнее учит, чем больше внимания она уделяет исследованию начал. Ведь обучение - это указание причины для каждой вещи. Схема "знание ради знания" присуща науке о том, что наиболее достойно познания. Такая наука совершенна. А наиболее достойны познания причины и начала нашего мира, то есть его аксиомы. Через них, на их основе познаётся всё остальное. Получается, самая главная наука - та, которая познаёт цель, ради которой надлежит действовать. В каждом отдельном случае эта цель - то или иное благо, а в общем случае - благо вообще.

Из всего сказанного следует, что мудрость - это наука, исследующая <b>самые базисные</b> причины и начала. Ведь и благо - цель существования - это тоже причина.
\footnotemark[2]
Но что это за наука?

Мудрость - чисто теоретическая наука. Это понимают даже первые философы. Философствовать людей всегда побуждает удивление. В начале мы удивляемся всему вообще. Затем это приедается, и мы продвигаемся дальше. Мы задаёмся более значительными вопросами: о смене положения Луны, Солнца, звёзд и о происхождении Вселенной. Человек удивляющийся считает себя незнающим. Получается, что люди изобрели философию, чтобы избавиться от удивления и, следовательно, от незнания. Мы стремимся к знанию ради самого этого знания; не для какой-либо пользы, а ради понимания и ради избавления от удивления. Сама история подтверждает это. Мы начали искать такое знание только когда нашли всё необходимое для жизни. Поэтому ясно, что мы ищем такое знание не ради какой-то надобности: у нас и так всё есть. Можно провести аналогию: свободным мы называем человека, который живёт ради самого себя. А наука мудрости - единственно свободная наука, ибо она одна существует ради самой себя.
\footnotemark[3]

Обладание мудростью кажется выше человеческих возможностей. Ведь во многих отношениях природа людей - рабская. Симонид говорил: "лишь бог мог бы иметь этот дар, а человеку не подобает искать несоразмерного ему знания". Наши поэты говорят, что бог будет завидовать нам, обладающим мудростью. Ведь мы будем уподобляться мудрому богу - поэтому излишне мудрые будут несчастны. Но бог не завистлив. Мудрость следует ценить больше всех других наук. Ибо наиболее божественная наука также и наиболее ценима. А таковой может быть только мудрость, но в двояком смысле. Во-первых, божественна та из наук, которой мог бы обладать бог. Во-вторых, божественна всякая наука о божественном. Только мудрости подходят два этих пункта. Ведь бог - это причина и начало, которые мудрость и исследует. Такой наукой мог бы обладать только бог. Таким образом, все другие науки более необходимы, нежели мудрость, но лучше - нет ни одной.

Как я отметил ранее, всё начинается с удивления, а кончается - утратой удивительности. При обретении мудрости отправная точка становится обыденностью. Ничему бы сильнее не удивился бы математик, как если бы диагональ квадрата оказалась бы рациональным числом.

Итак, сказано, какова природа искомой науки - мудрости и какова цель, к которой должны привести поиски оной.

\end{document}


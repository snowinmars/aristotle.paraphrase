\documentclass{article}

\usepackage[T2A]{fontenc}
\usepackage[utf8]{inputenc}
\usepackage[russian]{babel}

\linespread{1.1}
\setlength{\parskip}{1em}
\usepackage[left=1cm,right=1cm,top=1cm,bottom=2cm]{geometry}

\begin{document}

Итак, надо понять начала и причины. Тогда мы поймём всё, что из них следует.

Таких начал четыре
\footnotemark[1]
:<ul><li>Сущность вещи - суть её бытия. К этой сути обращены все вопросы "почему". Когда мы найдём первое "почему" - мы найдём причину и начало.</li><li>Материя вещи.</li><li>Причина появления.</li><li>Цель - "то, ради чего". Отмечу, что цель существования - всегда благо.</li></ul>

Я в достаточной мере рассмотрел эти начала в сочинении "О природе", но для цельности изложения повторю рассуждения.

Большинство первых философов считает началами лишь материальные начала. Из какой материи состоит объект, из какой возникает и в какую, погибая, превращается. Они также утверждают, что в такой цепочке трансформаций сама вещь остаётся, но изменяется в своих проявлениях. Иными словами, они утверждают, что при трансформации вещи сохраняется некое её естество
\footnotemark[2]
. Например, Сократ в течении своей жизни изменился: из необразованного стал образованным. Мы же не говорим, что необразованный Сократ исчез, а появился другой, образованный. Сократ остаётся один и тот же: меняются его проявления. Так же, - говорят первые философы, - материя не возникает из ниоткуда и не исчезает в никуда.

Каждый из первых философов описывет эту идею по-своему
\footnotemark[3]
. Фалес
\footnotemark[4]
- автор идеи о том, что материя не исчезает - утверждет, что начало - вода. Кто-то считает, что древнейшие люди, жившие задолго до нас и первыми писавшие о богах, держались того же взгляда на природу. Что касается Гиппона, то мы пропустим его рассуждения ввиду скудности его мыслей. Анаксимен же и Диоген считают, что воздух первее воды. Гиппас из Метапонта и Гераклит из Эфеса считают, что огонь первее. Эмпедокл же считает началами четыре элемента: вода, воздух, огонь и земля. А Анаксагор из Клазомен утверждает, что начал бесконечно много.

Исходя из этого, причину существования любой вещи можно было бы считать чисто материальной. Но такое объяснение совершенно неудовлетворительно. Действительно, пусть всякое возникновение и уничтожение исходит из нескольких материальных начал. Но почему это происходит? Что причина этого? Ведь не сама же материя вызывает собственную перемену. Не дерево и не медь меняют себя. Не дерево делает ложе, и не медь - изваяние. Что-то другое является причиной изменения. А искать эту причину - значит искать некую иную, нематериальную причину. В моих терминах, эта причина и есть "причина появления".

Итак, часть философов считает, что начало вещи - это какой-то один неизменный элемент, например, огонь. В конце концов, такие философы объявляют вечной и неизменной не только материю, но и всю природу целиком. Ведь им не удаётся найти эту причину изменений. Разве что Парменид несколько преуспел в этом, да и то поскольку он полагает (в некотором смысле) два вида причин.

Лучше дела идут у тех, кто признаёт началами несколько элементов. Один элемент - например, огонь - они рассматривают как двигатель изменений, а другой элемент - например, воду - как тормоз изменений.

Но на основе этих теорий мы всё равно не можем понять природу. Сама истина побуждает искать дальше. Естественно, ни огонь, ни вода, ни что-то такое не может быть причиной существования или изменения качеств объекта. Впрочем, первые философы так и не думают. Но столь же неверно считать, что всё происходит само собой. То ли Анаксагор, то ли Гермотим из Клазомен, то ли ещё кто-то говорил следующее: "Ум находится как в живых существах, так и в природе. Этот ум и есть причина миропорядка". Это намного лучше необдуманных рассуждений предшественников. Кто придерживается такого взгляда - тот признаёт идею "совершенства" в вещах и началом, и причиной изменений. Из этой идеи выросло понятие эйдоса, о котором будет сказано ниже.

\end{document}


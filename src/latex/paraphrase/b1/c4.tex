\documentclass{article}

\usepackage[T2A]{fontenc}
\usepackage[utf8]{inputenc}
\usepackage[russian]{babel}

\linespread{1.1}
\setlength{\parskip}{1em}
\usepackage[left=1cm,right=1cm,top=1cm,bottom=2cm]{geometry}

\begin{document}

Можно предположить, что Гесиод первым обратился к "совершенству" как к началу и причине существования вещи
\footnotemark[1]
. Он считет любовь или вожделение началом и пишет: "Прежде всего во Вселенной Хаос зародился, а следом - широкогрудая Гея". Или, например, Парменид замечает, что первее всех богов Вселенная замыслила Эрота. А кто-то считает первоначалами дружбу и вражду. Потому что в природе явно есть противоположность благу. Не только красота, но и уродство. Притом в природе плохого больше, чем хорошего. И потому можно сказать, что Эмпедокл первым говорит о зле и благе как о началах. Но с оговоркой: только если причина всех благ - само благо, а причина всех зол - само зло.

Упомянутые философы касаются только двух начал из четырёх перечисленных в третьей главе: материи и причины появления вещи. К тому же, философы касаются их нечётко и неуверенно. Так сражаются необученные: поворачиваясь во все стороны, они наносят иногда удачные удары, но не со знанием дела. Точно так же эти философы не до конца понимают, что говорят. Ведь далее они почти не прибегают к своим началам.

Анаксагор рассматривает ум как причину изменения вещей. Но когда ему сложно объяснить, почему что-то существует, он ссылается на ум. А в остальных случаях объявляет причиной что угодно, но не ум.

Эмпедокл считает, что причин две: дружба и вражда.
\footnotemark[2]
Он прибегает к своим причинам в большей мере, но всё равно недостаточно. Вдобавок у него отсутствует самосогласованность: дружба часто разделяет, а вражда - соединяет. Когда мировое целое через вражду распадается на элементы, огонь через вражду соединяется в одно.

Но Эмпедокл впервые разделил одну движущую причину на две, притом противоположенные. Кроме того, его четыре элемента (огонь, земля, воздух и вода) впервые формируют диалектическую пару "огонь" и "не-огонь" ("земля-воздух-вода"). Такой вывод можно сделать, изучая его стихи.

А Левкипп и его последователь Демокрит признают элементами полноту и пустоту. Одно называют сущим, другое - не-сущим. Например, полное и плотное - сущим, а пустое и разреженное - не-сущим. Поэтому они и говорят, что сущее существует не в большей степени, чем не-сущее. Ведь и не-пустота, и пустота состоят из материи (полной и пустой соответственно). Материальной причиной существующего они называют обе эти противоположности. Следуя учению атомистов, Левкипп и Демокрит утверждают, что отличия атомов (у них их, правда, всего два: атом пустоты и атом полноты) - это причины всех остальных отличий. А отличия атомов бывают трёх видов:<ul><li>очертания - как "А" от "Р",</li><li>порядок - как "АР" от "РА",</li><li>положение - как "Ь" от "Р".</li></ul> А вопрос о причине изменений атомов они, подобно остальным, легкомысленно обошли.

[ref:origin 4]
Итак, на этом я завершу обзор исследований причин моих предшественников.

\end{document}


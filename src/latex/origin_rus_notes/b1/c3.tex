\documentclass{article}

\usepackage[T2A]{fontenc}
\usepackage[utf8]{inputenc}
\usepackage[russian]{babel}

\linespread{1.1}
\setlength{\parskip}{1em}
\usepackage[left=1cm,right=1cm,top=1cm,bottom=2cm]{geometry}

\begin{document}


\footnotetext[1]{Аристотель часто заменяет положительную формулировку понятия неопределенным обозначением его как ответа на тот или иной вопрос. Так, например, — «то, откуда начало движения»  (οθενή αρχή τής κινήσεως), — источник движения; «то, ради чего» (τό ού ενεκα), — цель. Подобный же характер носит его знаменитая формула для обозначения логической сущности: «что есть (или—что представляет собою, дословно — чем было) бытие (для такой-то вещи) (τό τί ήν είναι)» — сущность бытия вещи. Я перевожу эту формулу техническим выражением «суть бытия».}


\footnotetext[2]{Как отмечает Ross,— «взятое в конечном счете основание, почему (вещь такова, как она есть)» и «то основное, благодаря чему (вещь именно такова)», весьма неуклюжим образом указывают на одно и то же («формальное») начало.}


\footnotetext[3]{Точнее — начала, относящиеся к разряду материи. Eidos, о котором здесь говорится, не имеет в  данном  случае, разумеется, обычного у Аристотеля технического значения — «формы», но должно переводиться как «вид» (логический) или «разряд», «группа»: началами всех вещей у первых философов являются лишь те, которые принадлежат к виду или разряду материи (т.е. к группе начал материальных).}


\footnotetext[4]{У Аристотеля — «музический» (μουσικός), — постоянный пример для качества. В этом употреблении, может быть, точнее всего будет передача «образованный», в противоположность αμουσος — необразованный, грубый; для «музыки» (μουσική), вернее «музического искусства», у Платона — обширный ряд оттенков от поэзии до философии (Phaed. 60 D-61 А).}


\footnotetext[5]{Так передает здесь φύσις (дословно — «природа») Маковельский; по смыслу можно перевести «от природы существующее бытие».}


\footnotetext[6]{Дословно: «отцами».}


\footnotetext[7]{Гомер говорит: «Океана источник богов и мать их Тефиду» (Илиада, XV, 201, 246), причем, однако, под богами надлежит здесь разуметь не богов вообще, коих Океан и Тефида родителями не были, а женские морские божества — Океанид. Кого Аристотель имеет здесь в виду, кроме Гомера, установить трудно. По мнению Alex., речь может еще идти о Гесиоде. Есть также некоторые основания относить это указание Аристотеля к космогониям так называемых орфиков, возводивших свои учения к мифическому певцу Орфею.}


\footnotetext[8]{О Гиппоне, жившем в эпоху Перикла, Аристотель дает резко отрицательный отзыв еще и в другом месте (de anima 1, 2, 405, b 2). Необходимо, однако, отметить, что Гиппон был убежденным материалистом, не признававшим никакого бытия помимо чувственно воспринимаемых тел, вследствие чего он и получил прозвание безбожника (Alex. Comm. 428, 21—23). Отсюда, может быть, и антипатия Аристотеля.}


\footnotetext[9]{О возникновении и уничтожении элементов по Эмпедоклу можно говорить лишь в том смысле, что отдельные из них накапливаются иногда в большом, иногда в малом количестве на, пути к объединению по отдельным элементам (под влиянием действия вражды); и затем рассеиваются из этого объединения (см. Arlst Met. А 4,985 а 25—28 и Ross- I 181).}


\footnotetext[10]{Здесь говорится о принимавшихся Анаксагором элементарных предметах с однородными друг другу частями (у Анаксагора они назывались семенами вещей, а позже у Аристотеля получили название «гомеомерных», — подобночастных — вещей, или гомеомерий). Аристотель в нашем месте иллюстрирует их примером огня и воды, так как сам он, в отличие от Анаксагора, также видел в этих последних вещи с однородными частями. Для анаксагоровских гомеомерий устанавливается один путь генезиса и уничтожения — через соединение и разделение; ограничение «почти» вводится Аристотелем с своей точки зрения потому, что возникновение через соединение однородного как раз не относится к обычным (эмпедокловским) элементам, которые Анаксагор считал не за элементы, но за смеси из всех первичных «семян» (Ср. Ross I 132—133 и цитируемое им место Arist. De Coelo Г 3, 302 α 28 сл).}


\footnotetext[11]{Т.е. причину движения.}


\footnotetext[12]{αίτίαν του καλώς) — дословно: причину «того, что <вещи находятся> в хорошем состоянии».}


\end{document}


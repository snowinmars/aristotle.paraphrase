\documentclass{article}

\usepackage[T2A]{fontenc}
\usepackage[utf8]{inputenc}
\usepackage[russian]{babel}

\linespread{1.1}
\setlength{\parskip}{1em}
\usepackage[left=1cm,right=1cm,top=1cm,bottom=2cm]{geometry}

\begin{document}


\footnotetext[1]{ См. «Физика» II 3, 194 Ь 16 --- 195 Ь 30. }


\footnotetext[2]{ Речь, возможно, идет об «апейроне» (неопределенном) Анаксимандра, хотя, насколько известно, сам он не высказывался о ближайших качествах принятой им первоосновы сущего.}


\footnotetext[4]{ Те - Платон и его школа. Т. е. считают единое или сущее благом, как разбиравшиеся перед тем философы считали благом дружбу и ум.}


\footnotetext[5]{ Конъектура Bywater'a --- Ross'a: toioyton вместо toyton (этих).}

\end{document}


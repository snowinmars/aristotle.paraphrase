\documentclass{article}

\usepackage[T2A]{fontenc}
\usepackage[utf8]{inputenc}
\usepackage[russian]{babel}

\linespread{1.1}
\setlength{\parskip}{1em}
\usepackage[left=1cm,right=1cm,top=1cm,bottom=2cm]{geometry}

\begin{document}


\footnotetext[1]{Дословно: «много сходств».}


\footnotetext[2]{Т.е. выражение свойств и отношений, присущих гармоническим сочетаниям.}


\footnotetext[3]{Pragmateia = и усиленная работа и получающееся в результате произведение.}


\footnotetext[4]{Вымышленное тело, которое пифагорейцы помещали между землею и также ими вымышленным центральным огнем. С земли противоземля не видна, как не видно и центрального огня, потому что люди находятся только на внешней части земли, никогда не поворачивающейся к мировому центру.}


\footnotetext[5]{По указанию Alex. (Comm.), имеются в виду сочинение «О небе» (II 13) и утраченная работа «Взгляды пифагорейцев (αί των Πυθαγορικών δόςα$).}


\footnotetext[6]{По объяснению Ross'a Аристотель хочет сказать, что пифагорейцы приписывали числу характер материальной причины и в то же время делали первые попытки понять его в качестве причины формальной, так что έςεις есть достаточно естественный эквивалент для εΐδη (Ross I, 147.)}


\footnotetext[7]{Последнее противопоставление обычно толкуется как противопоставление равностороннего четыреугольника (специально — квадрата) разностороннему прямоугольнику.}


\footnotetext[8]{Bonitz удачно переводит: «... был младшим современником Пифагора»... Может быть, дословно можно было бы передать: «... пришелся на старые годы II.». Diels предлагает — νέος έπι γέρ. Πυθ: «был молодым, когда Пифагор был стариком».}


\footnotetext[9]{Мы бы сказали: имманентных}


\footnotetext[10]{Или: «которые высказались о совокупности бытия, что это — единая природа...»}


\footnotetext[11]{Аристотель хочет указать, что теории, созданные относительно природы сущего различными философами Элейской школы, различались между собой как в отношении связности (и, следовательно, убедительности) аргументации, так и относительно характера, приписываемого ими истинному бытию (Bon. Comm. 83).}


\footnotetext[12]{Дословно: «речь о них не стоит никоим образом в связи с теперешним рассмотрением причин».}


\footnotetext[13]{Аристотель говорит о «физиологах», употребляя это слово в непосредственном смысле и называя так мыслителей, в общей форме рассуждавших о природе («физиология» — философия природы, натурфилософия).}


\footnotetext[14]{Т.е. ни логической, ни материальной.}


\footnotetext[15]{У Аристотеля — «у них». Может быть «у этих <двух> мыслителей», как принято в русском переводе. Ross объясняет: «у этих начал», хотя здесь как раз трудно говорить о началах, потому что в Элейской философии по Аристотелю бытие именно не выступает как «причина».}


\footnotetext[16]{В греческом тексте здесь (а 3 — 8) фактически — значительно более сложная конструкция, характер которой в общих чертах можно бы было передать следующим образом: «<получили мы это> от наиболее ранних из них, которые полагали, что начало является телесным..., причем некоторые принимали, что телесное начало одно, а другие, что таких начал несколько, но те и другие отнесли их в разряд материи; а также — от некоторых таких философов, которые принимали и эту причину, и кроме нее — ту...» (см. Ross I 155 к строкам а 4—9).}


\footnotetext[17]{Аристотель не хочет сказать, что пифагорейцы признали те же два начала, как «физиологи», принимавшие материю и причину движения, а только, что они также принимали в общем два начала (здесь это будут — материя и принцип формы).}


\footnotetext[18]{Т.е. не составляют лишь свойства или определения отличных от них по существу стихий.}


\footnotetext[19]{Т.е. чему даются такие определения (что определяется как сущее и единое).}


\footnotetext[20]{Относительно «того, что есть вещь» (περι του τί εστίν), — здесь разумеется ответ на вопрос о существе вещи, ср. выше примечание 1 к 3-й главе.}


\footnotetext[21]{По мнению Ross'a (см. I, 156—157), Аристотель отмечает «упрощенность» использования формальной причины у пифагорейцев в двух отношениях. С одной стороны, даваемые ими определения были поверхностны, они выхватывали какие-нибудь внешние признаки определяемого предмета, например (Alex. Comm. 36,26), существо дружбы видели в воздаянии равным за равное, или существо справедливости — в испытании равного содеянному, хотя это только — отдельные моменты в понятии дружбы или справедливости. С другой стороны, принятые ими определения они сближали с первым обладавшим аналогичными свойствами числом, хотя этими свойствами обладал и ряд последующих чисел, а потому между их определениями и указывавшимися для выражения сущности определяемого предмета числами тоже не было полного соответствия (обратимости). Так, например, произведение равного на равное (ίσάχις ϊσον), по их мнению, находило себе выражение в числе 4, а потому они видели в числе 4 сущность всего, что характеризовалось как сочетание равного с равным (дружба, справедливость и т.д.): получались такие же последствия, как если бы мы стали считать, что «двойное» исчерпывается числом два, между тем как на самом деле два есть лишь первый случай «двойного», и двойное применимо также к четырем, к шести и т.д.; при таком ходе рассуждения, к одному и тому же — к двум — должны бы были у пифагорейцев свестись и эти дальнейшие числа, и все те вещи, к которым применимо отождествленное с числом два определение «двойное».}


\footnotetext[22]{Ross поясняет — «более поздних в доплатоновском периоде».}


\end{document}


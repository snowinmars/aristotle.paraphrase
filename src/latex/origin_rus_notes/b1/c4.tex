\documentclass{article}

\usepackage[T2A]{fontenc}
\usepackage[utf8]{inputenc}
\usepackage[russian]{babel}

\linespread{1.1}
\setlength{\parskip}{1em}
\usepackage[left=1cm,right=1cm,top=1cm,bottom=2cm]{geometry}

\begin{document}


\footnotetext[1]{«Теогония» Гесиода, стихи 116—120.}


\footnotetext[2]{Или — «выставлял первую как источник положительных, вторую — как источник отрицательных свойств.}


\footnotetext[3]{Дословно: «скорее все выставляет причиной происходящих вещей, нежели ум».}


\footnotetext[4]{Согласно чтению Ross'а — τό τήν αίτίαν διελεΐν, вместо прежде читавшегося ταύτην τήν αίτίαν διελών (так как в окружающем тексте не к чему относить ταύτην).}


\footnotetext[5]{Слова «и разреженное» одна из главных рукописей (A^b) выпускает, и их есть основание рассматривать как раннюю вставку.}


\footnotetext[6]{Ross указывает, что ρυσμός есть правильная ионическая форма вместо ρυθμός, и приводит ряд параллельных мест, где ρυθμός употреблено в смысле «форма» (в том числе у Геродота и Ксенофонта).}


\footnotetext[7]{В примере, приводимом Аристотелем, греческие буквы указываются так: «А отличается от N формой, ΑΝ от ΝΑ порядком, Η (современный Аристотелю знак для Ζ) от Η положением» (Ross, 1,141).}


\end{document}


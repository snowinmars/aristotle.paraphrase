\documentclass{article}

\usepackage[T2A]{fontenc}
\usepackage[utf8]{inputenc}
\usepackage[russian]{babel}

\linespread{1.1}
\setlength{\parskip}{1em}
\usepackage[left=1cm,right=1cm,top=1cm,bottom=2cm]{geometry}

\begin{document}

Имея в виду подвергнуть крптике учение Платона, Аристотель отмечает, что гипотеза, устанавливающая идеи, опрометчиво удваивает число вещей, которые надо объяснить; затем он в первую очередь разбирает аргументы, которые Платон выдвигал в пользу существования идеи·, и констатирует в них различные логические дефекты: в одних вывод не следует с (силлогистическою) необходимостью, другие идут дальше, чем хотел сам философ. Далее он разбирает взаимоотношение между идеями и чувственнымн вещами и вопрос о том, какое воздействие производят идеи на чувственные вещи. После этого он переходит к природе чисел (к которым Платон пытался свести идеи) и излагает те трудности, которые в этом случае получаются как при попытках формулировать существо идей --- чисел, так и при объяснении из них вещей. В результате Аристотель делает вывод, что в платоновской школе весь вопрос был поставлен самым превратным образом: были отвергнуты вещи и причины, непосредственно данные, и философы установили сокровенные и совершенно бесплодные начала; их доказательства не доказывают, чего они хотят, и природу геометрических объектов нельзя совместить с характером всего учения в целом. Наконец, в виду того, что они установили одни и те же начала для всех получается, что их нельзя ни познавать, ни считать прирожденными человеку от природы

\end{document}


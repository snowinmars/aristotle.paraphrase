\documentclass{article}

\usepackage[T2A]{fontenc}
\usepackage[utf8]{inputenc}
\usepackage[russian]{babel}

\linespread{1.1}
\setlength{\parskip}{1em}
\usepackage[left=1cm,right=1cm,top=1cm,bottom=2cm]{geometry}

\begin{document}

\footnotemark[1]{ Здесь и несколько глав далее Аристотель активно описывает взгляды современников на физику. Я сокращу такие части: детальные взгляды греческих философов на физику уже не актуальны и, признаться, не особо понятны.}

\footnotemark[2]{ Есть хороший пример философии Пифагора в современном кино: фильм "Пи" (1997). <br /> <iframe width="560" height="315" src="https://www.youtube.com/embed/TxCt093lDWw" frameborder="0" allow="accelerometer; autoplay; encrypted-media; gyroscope; picture-in-picture" allowfullscreen></iframe>}

[ref:paraphrase 3]
\footnotemark[3]{ Ранее по тексту термин "единое" не разъясняется. О нём будет много сказано в пятой книге. Пока можно считать "единое" самой высшей сущностью - богом?}

\end{document}


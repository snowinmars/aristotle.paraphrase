\documentclass{article}

\usepackage[T2A]{fontenc}
\usepackage[utf8]{inputenc}
\usepackage[russian]{babel}

\linespread{1.1}
\setlength{\parskip}{1em}
\usepackage[left=1cm,right=1cm,top=1cm,bottom=2cm]{geometry}

\begin{document}

\footnotemark[1]{ Для нас сейчас - это кварки.}

\footnotemark[2]{ Напомню, что говорится об Эмпедокле в третьей главе. Он выделяет четыре элемента: воду, воздух, огонь и землю; а причиной изменений вещи считает диалектическую пару "дружба-вражда".}

\footnotemark[3]{ Напомню, что говорится об Анаксагоре в третьей главе. Он считает, что элементов бесконечно много, а причиной изменений считает ум.}

\end{document}


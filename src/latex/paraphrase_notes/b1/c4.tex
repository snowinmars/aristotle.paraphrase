\documentclass{article}

\usepackage[T2A]{fontenc}
\usepackage[utf8]{inputenc}
\usepackage[russian]{babel}

\linespread{1.1}
\setlength{\parskip}{1em}
\usepackage[left=1cm,right=1cm,top=1cm,bottom=2cm]{geometry}

\begin{document}

\footnotemark[1]{ Здесь и несколько глав далее Аристотель активно описывает взгляды современников на физику. Я сокращу такие части: детальные взгляды греческих философов на физику уже не актуальны и, признаться, не особо понятны.}

\footnotemark[2]{ Эмпедокл выделяет четыре "атома" или элемента: воздух, воду, огонь и землю. А причиной изменений вещей, состоящих из этих элементов - дружбу и вражду.}

\end{document}


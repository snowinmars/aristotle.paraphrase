\documentclass{article}

\usepackage[T2A]{fontenc}
\usepackage[utf8]{inputenc}
\usepackage[russian]{babel}

\linespread{1.1}
\setlength{\parskip}{1em}
\usepackage[left=1cm,right=1cm,top=1cm,bottom=2cm]{geometry}

\begin{document}

Нам очевидным образом надлежит достигнуть знания о первоначальных причинах, --- мы ведь тогда приписываем себе знание каждой вещи, когда, по нашему мнению, мы постигаем первую причину.
\footnotemark[1]
А о причинах речь может итти в четырех смыслах: одной такой причиной мы признаем сущность и суть бытия
\footnotemark[2]
(«основание, почему»
\footnotemark[20]
(вещь такова, как она есть), восходит в конечном счете к понятию вещи, а то основное, благодаря чему (вещь именно такова), есть некоторая причина и начало);
\footnotemark[21]
другой причиной мы считаем материю и лежащий в основе субстрат;
\footnotemark[3]
третьей --- то, откуда идет начало движения; четвертой --- причину, противолежащую (только что) названной, а именно --- «то, ради чего» (существует вещь), и благо (ибо благо есть цель всего возникновения и движения). Вопрос об этих причинах был, правда, в достаточной степени рассмотрен у нас в книгах о природе,
\footnotemark[18]
но все же привлечем также и тех, которые раньше нас обратились к исследованию вещей и вели философские рассуждения об истинном бытии. Ибо очевидно, что и они указывают некоторые начала и причины. Поэтому если мы переберем их учения, (от этого) будет некоторая польза для теперешнего исследования: или мы найдем какой-нибудь другой род причин, или больше будем.верить тем, которые ука зываются в настоящее время.

Из тех, кто первые занялись философией, большинство считало началом всех вещей одни лишь начала в виде3  материи: то, из чего состоят все вещи, из чего первого они возникают и во что в конечном счете разрушаются, причем основное существо пре бывает, а по свойствам своим меняется, --- это они считают элементом и это --- началом вещей. И вследствие этого они полагают, что ничто не возникает и не погибает, так как подобная основная природа
\footnotemark[4]
всегда сохраняется, подобно тому, как и про Сократа мы не говорим ни --- что он становится просто, когда он становится прекрасным или образованным
\footnotemark[5]
, ни --- что он погибает, когда он утрачивает эти свойства, ввиду того, что пребывает лежащий в основе субстрат --- сам Сократ. Таким же образом (не допускают они возникновения и погибели) и для всего остального; ибо должно быть некоторое природное естество
\footnotemark[19]
 --- или одно, или больше, чем одно, откуда возникает все остальное, причем само это естество остается в сохранности.

Количество и форму для такого начала не указывают все одинаково, но Фалес --- родоначальник такого рода философии --- считает его водою (вследствие чего он и высказывал мнение, что земля находится на воде); к этому предположению он, можно думать, пришел, видя, что пища всех существ --- влажная и что само тепло из влажности получается и ею живет (а то, из чего (все) возникает, это и есть начало всего). Таким образом он отсюда пришел к своему предположению, а также потому, что семена всего (что есть) имеют влажную природу, а у влажных вещей началом их природы является вода.

Есть и такие, которые полагают, что и (мыслители) очень древние, жившие задолго до теперешнего поколения ж впервые занявшиеся теологией, держались именно таких взглядов относительно природы: Океана и Тефиду
\footnotemark[6]
они сделали источниками6  возникновения7 , и клятвою Гюгов стала у них вода, а именно Стикс, как они его называли; ибо почтеннее всего --- самое старое, а клятва, это --- самое почтенное. Во всяком случае, является ли это мнение о природе древним и давнишним, Что, может быть, и недостоверно, но о Фалесо говорят, что он так высказался относительно первой причины (что касается Гипиона,
\footnotemark[7]
его, пожалуй, не всякий согласится поставить рядом с этими философами вследствие ограниченности его мысли).

С другой стороны, Анаксимен и Диоген
\footnotemark[8]
ставят воздух раньше, нежели воду, и из простых
\footnotemark[10]
тел его главным образом принимают за начало; Гиппас из Метапонта
\footnotemark[11]
и Гераклит из Эфеса (выдвигают) огонь, Эмпедокл --- (известные) четыре элемента, к тем, которые были названы,
\footnotemark[12]
на четвертом месте присоединяя землю; элементы эти всегда пребывают, и возникновение
\footnotemark[13]
для них обозначает только (появление их) в большом и в малом
\footnotemark[9]
числе в то время, когда они собираются (каждый) в одно и рассеиваются из одного.

Анаксагор из Клазомон, будучи по возрасту раньше этого последнего, а по делам своим позже его, утверждает, что начала не ограничены (по числу): по его словам, почти все подобночастные
\footnotemark[14]
предметы, являющиеся таковыми по образцу воды или огня10 , возникают и уничтожаются именно таким путем
\footnotemark[14]
 --- только через соединение и разделение, а иначе не возникают и не уничтожаются, но пребывают вечно.

Исходя из этих данных, за единственную причину можно было бы принять ту, которая указывается в виде материи. Но по мере того как они в этом направлении продвигались вперед, самое положение дела указало нм путь и со своей стороны принудило их к (дальнейшему) исследованию. В самом деле, пусть всякое возникновение и уничтожение сколько угодно происходит на основе какого-нибудь одного или хотя бы нескольких начал, почему оно происходит, и что --- причина этого? Ведь не сам лежащий в основе субстрат производит перемену в себе, например ни дерево, ни медь (сами не являются причиной, почему изменяется каждое из них, и не производит дерево --- кровать, а медь --- статую, ио нечто другое составляет причину (происходящего) изменения. А искать эту причину --- значит искать другое начало, как мы бы сказали --- то, откуда начало движения. Те, которые с самого начала взялись за подобное исследование и утверждали единство лежащего в основе субстрата, не испытывали никакого недовольства собой, но, правда, некоторые из (таких) сторонников единства, как бы под давлением этого исследования, объявляют единое неподвижным, как равно и всю природу, не только в отношепип возникновения и уничтожения (это --- учение старинное, и все с ним соглашались), но и в отношении всего остального изменения; и это их своеобразная черта. Из тех, таким образом, кто объявлял мировое целое единым, никому не довелось усмотреть указанную (сейчас) причину,
\footnotemark[15]
разве только Пармениду, да и этому последнему --- постольку, поскольку он полагает не только существование единого, но --- в известном смысле --- и существование двух причин.
\footnotemark[16]
Тем же, кто вводит множественность (начал), скорее можно говорить (о такой причине), например тем, кто принимает теплое и холодное, или огонь и землю: они пользуются огнем, как обладающим двигательного природой, а водою, землей и тому подобными (элементами) на противоположный лад.

После этих философов и такого рода начал, так как эти последние были недостаточны, чтобы вывести из них природу вещей, стали, снова побуждаемые --- как мы сказали --- самой истиной, искать следующего затем начала. Что одни вещи находятся в хорошем и прекрасном состоянии, а другие приходят к нему в процессе своеговозникновения, причиной этого не подобает быть ни огню, ни земле, ни чему-либо другому в этом роде, и те философы подобного  взгляда наверно и не держались; а с другой стороны, не хорошо было также вверять такое дело случаю и самопроизвольному процессу. Поэтому тот, кто сказал, что разум находится, подобно тому как в живых существах, также и в природе, и что это он --- виновник благоустройства мира и всего мирового порядка, этот человек представился словно трезвый по сравнению с пусто словием тех, кто выступал раньше. Явным образом, как мы знаем, взялся за такт* объяснения Анаксагор, но есть указание, что прежде об этом сказал Гермотим из Клазомен.
\footnotemark[17]
Те, которые стояли на этой точке зрения, в то же время признали причину совершенства12  (в вещах) первоначалом вещей, и притом --- таким, от которого вещи имеют движение.

\end{document}


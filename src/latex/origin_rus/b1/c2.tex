\documentclass{article}

\usepackage[T2A]{fontenc}
\usepackage[utf8]{inputenc}
\usepackage[russian]{babel}

\linespread{1.1}
\setlength{\parskip}{1em}
\usepackage[left=1cm,right=1cm,top=1cm,bottom=2cm]{geometry}

\begin{document}

Так как эта именно наука является целью наших поисков, то нужно рассмотреть, какими именно причинами и какими началами должна заниматься наука, которая носит название1  мудрости. Если учесть те мнения, которые у нас есть о мудром человеке, то отсюда вопрос, может быть, выяснился бы больше. Мы предполагаем прежде всего, что мудрый знает все, насколько это возможно, не имея знания в отдельности о каждом предмете. Далее, мы считаем мудрым того, кто в состоянии узнать вещи трудные и не легко постижимые для человека (ведь чувственное восприятие общо всем, а потому это — вещь легкая, и мудрости (в нем) нет никакой). Кроме того, более мудрым во всякой науке является человек более точный и более способный научать, а из наук в большей мере считается мудростью та, которая избирается ради нее самой и в целях познания, а не та, которая привлекает из-за ее последствий и та, которая играет более руководящую роль — в большей мере, чем та, которая занимает служебное положение; ибо мудрому но надлежит получать распоряжения, по давать их, и не он должен повиноваться другому, а ему — тот, кто менее мудр.

Такие мнения и столько их имеем мы о мудрости и мудрых людях. Из перечисленных свойств знани е обо всем должно быть у того, кто в наибольшей мере владеет знанием в общей форме: такому человеку некоторым образом
\footnotemark[1]
известна вся совокупность вещей (которая входит в круг этого знания).
\footnotemark[7]
Можно сказать, что и наиболее трудны для человеческого познанпя такие начала — начала наиболее общие: они дальше всего от чувственных восприятий. А наиболее точными являются те из наук, которые больше всего имеют дело с первыми началами: те, которые исходят от меньшего числа (элементов), более точны, нежели те, которые получаются в результате прибавления (новых свойств), например арифметика точнее геометрии.
\footnotemark[2]
Но и обучать более пригодна та наука, которая рассматривает при чины; ибо научают те люди, которые указывают причины для каждой вещи. А знание и понимание, происходящие ради них самих, более всего свойственны науке о предмете, познаваемом в наибольшей мере:
\footnotemark[7]
тот, кто отдает предпочтение знанию ради знания, больше всего отдаст предпочтения науке наиболее совершенной,
\footnotemark[8]
а это—наука о максимально познаваемом предмете. Обладают же такою познаваемостью первые элементы и причины, ибо с помощью их и на их основе познается все остальное, а не они через то, что лежит под ними.
\footnotemark[5]
И наиболее руководящей из всех наук, и в большей мере руководящей, чем (всякая) наука служебная, является та, которая познает, ради чего надлежит делать каждую вещь; а такою конечною целью в каждом случае является благо и вообще наилучшее во всей природе.

Все указанные признаки требуют отнести обсуждаемое название к одной и той же науке: это должна быть наука, занимающаяся рассмотрением первых начал и причин; ведь и благо и «то, ради чего»
\footnotemark[3]
также является одною из причин. А что это наука не действенная,
\footnotemark[4]
ясно и (если судить) по людям, которые первые занялись философией. Ибо вследствие удивления люди и теперь и впервые начали философствовать, причем вначале они испытали изумление по поводу тех затруднительных вещей, которые были непосредственно перед ними, а затем понемногу продвинулись на этом пути дальше и осознали трудности в более крупных вопросах, например относительно изменений8  луны и тех, которые касаются солнца и звезд, а также — относительно возникновения мира.
\footnotemark[6]
Но тот, кто испытывает недоумение и изумление, считает себя незнающим (поэтому и человек, который любит мифы, является до некоторой степени философом, ибо миф слагается из вещей, вызывающих удивление). Если таким образом начали философствовать, убегая от незнания, то, очевидно, к знанию стали стремиться ради постижения (вещей), а не для какого-либо пользования (ими). То, что произошло на деле, подтверждает это: когда оказалось налицо почти все необходимое и также то, что служит для облегчения жизни и препровождения времени, тогда (только) стало предметом поисков такого рода разумное мышление. Ясно поэтому, что мы не ищем его ни для какой другой нужды. Но как свободный человек, говорим мы, это — тот, который существует ради себя, а не ради другого, так ищем мы и эту науку, так как она одна только свободна изо (всех) наук: она одна существует ради самой себя.

Поэтому и достижение ее по справедливости можно бы считать но человеческим делом; ибо во многих отношениях являет природа людей рабские черты10 , так что, по словам Симонида11 , «бог один иметь лишь мог бы этот дар», человеку же подобает искать соразмерного ому знания. Если поэтому слова поэтов чего-нибудь стоят, и божеской природе свойственна зависть, естественнее всего ей проявляться в этом случае, и несчастны должны бы быть все, кто хочет слишком многого. Но не может этого быть, чтобы божественное существо было проникнуто завистью,— напротив, и по пословице «лгут много поэты», — и не следует (какую-либо) другую науку считать более ценною, чем эту. В самом деле, наука наиболее божественная является и наиболее ценною. А божественною может считаться только одна эта — с двух точек зрения. На самом деле, божественною является та из наук, которою скорее всего мог бы владеть бог, и точно так же—если есть какая-нибудь наука о божественных предметах. А только к одной нашей науке подходит и то и другое. Бог по всеобщему мнению находится в числе причин и есть одно из начал, и такая наука могла бы быть или только у одного бога, или у бога в наибольшей мере. Таким образом, все науки более необходимы, нежели она, но лучше — нет ни одной.

Вместе с тем овладение этой наукой должно у нас известным образом привести к противоположным результатам по сравнению с первоначальными исканиями. Как мы говорили, все начинают с изумления, обстоит ли дело именно так: как (недоумевают), например, про загадочные самодвижущиеся игрушки, или (сходным образом) в отношении солнцеворотов, или несоизмеримости диагонали; ибо у всех, (кто еще не рассмотрел причину)  19 , вызывает удивление, если чего-нибудь нельзя измерить самою малою мерою. А под конец нужно притти к противоположному— и к лучшему, как говорится в пословице, — этим дело кончается и в приведенных случаях, когда в них разберутся: ведь ничему бы так не удивился человек сведущий в геометрии, чем если бы диагональ оказалась измеримой.

Теперь сказано, какова природа искомой науки и какова цель, к которой должно притти искание (этой науки) и все вообще исследование.

\end{document}


\documentclass[oneside, 17pt, dvipsnames]{extbook}

\title{
    Τὰ μετὰ τὰ φυσικά
    \\
    Βιβλία 1
}
\author{
    Аристотель
    \\
    Перевод Александра Владиславовича Кубицкого
}
\date{
    IV век до нашей эры
    \\
    Перевод 1934 года
}

\usepackage{xltxtra}
\usepackage{polyglossia}
\setdefaultlanguage{russian}
\setotherlanguages{english, greek}
\PolyglossiaSetup{russian}{indentfirst=true}
\PolyglossiaSetup{english}{indentfirst=true}
\PolyglossiaSetup{greek}{indentfirst=true}
\PolyglossiaSetup{latex}{indentfirst=true}
\usepackage{indentfirst} % отделять первую строку раздела абзацным отступом тоже

\usepackage{epigraph}
\setlength{\epigraphwidth}{1450pt}

% colors and fonts
\usepackage{xcolor}
\definecolor{bgColor}{RGB}{17,17,17}
\definecolor{fgColor}{RGB}{204,204,214}
\definecolor{maColor}{RGB}{123,204,123}
\setromanfont[Color=fgColor]{Liberation Sans}
\setsansfont{Liberation Sans}
\defaultfontfeatures{Ligatures={TeX},Renderer=Basic}
\pagecolor{bgColor}
\everymath{\color{maColor}}
\everydisplay{\color{fgColor}}

% geometry
\linespread{1.1}
\setlength{\parskip}{1em}
\setlength{\marginparpush}{30pt}
\usepackage[paperwidth=11in,paperheight=22in,landscape,left=1cm,right=1cm,top=1cm,bottom=2cm,outer=18cm,marginparwidth=15cm,marginparsep=1cm]{geometry}

% start a paragraph and a margin note at the same line
\newcommand{\alignedmarginpar}[1]{
  \leavevmode
  \marginpar{\footnotesize #1}
  \ignorespaces
}

\usepackage{hyperref}

\begin{document}

\maketitle
\tableofcontents

\section{Книга 1}

\epigraph{
Описывается постепенное восхождение от чувственного восприятия к познанию принципов. Общим у всех животных является восириятие через чувства. К чувственному восприятию у некоторых животных присоединяется сохранение воспринятых образов, или память. Ряд связанных между собой воспоминаний об одной и той же вещи создает у человека опыт. Из опыта возникает искусство и наука, причем искусство превосходит опыт тем, что его внимание направлено на общее в вещах, а не на отдельные вещи, и оно познает причину, вследствие чего тот, кто владеет искусством, способен обучать. Искусствам приписывается тем больше значения, чем более они далеки от непосредственных потребностей жизни; теоретические дисциплины выше практических, а самое высшее место занимает познание принципов, или мудрость.
}{Парафраз А.В. Кубицкого}

Все люди от природы стремятся к знанию. Свидетельством тому -- (наша) привязанность к чувственным восприятиям: помимо их пользы, восприятия эти ценятся ради них самих, и больше всех то из них, которое происходит с помощью глаз: ибо мы ставим зрение, можно сказать, выше всего остального, не только ради деятельности, но и тогда, когда не собираемся делать что-либо. Объясняется это тем, что чувство зрения в наибольшей мере содействует нашему познанию и обнаруживает много различий (в вещах).

Чувственным восприятием животные наделены от природы, а на почве чувственного восприятия у некоторых из них память не появляется, а у других она возникает. И животные, обладающие памятью, оказываются благодаря этому сообразительнее и восприимчивее к обучению, нежели те, у которых нет способности помнить; причем сообразительными, без обучения, являются все те, которые не могут слышать звуков, как, например, пчела, и если есть еще другая подобная порода животных; к обучению же способны те, которые помимо памяти обладают еще и чувством слуха.

\alignedmarginpar{$^1$ Пол — софист, ученик Горгия. Цитируемое утверждение П. вложено ему в уста в Платоновом «Горгии» (448 С) и находилось в составленном им сочинении (Gorg. 462 В).}
\alignedmarginpar{$^2$ Я перевожу согласно тексту Ross'a, который подчеркивает (I 117) указание Jacson'a, что под φλεγματώδης и χολώδης разумеются не больные, а здоровые с определенным темпераментом (флегматическим—холерическим), так что «страдающим такою-то болезнью» (χάμνουσιτ ήνδε τήν νοσον) в примере соответствует лишь в «сильной лихорадке» πυρέττουσι χαuτo.}
Все животные, (кроме человека), живут образами воображения и памяти, а опытом пользуются мало; человеческий же род прибегает также к искусству и рассуждениям. Появляется опыт у людей благодаря памяти: ряд воспоминаний об одном и том же предмете имеет в итоге значение одного опыта. И опыт представляется почти-что одинаковым с наукою и искусством. А наука и искусство получаются у людей благодаря опыту. Ибо опыт создал искусство, как говорит Пол, $^1$ —и правильно говорит,— а неопытность — случай. Появляется же искусство тогда, когда в результате ряда усмотрений опыта установится один общий взгляд относительно сходных предметов. Так, например, считать, что Каллию при такой-то болезни помогло такое-то средство и оно же помогло Сократу и также в отдельности многим, это—дело опыта; а считать, что это средство при такой-то болезни помогает всем подобным людям в пределах одного вида, например флегматикам или холерикам в сильной лихорадке $^2$, это — точка зрения искусства.

\alignedmarginpar{$^3$ Дословно: "случайным образом". По Аристотелевской формулировке врачующий излечивает не человека как такого, но больного, для которого быть человеком — случайное свойство, как и для человека быть больным — случайно.}
В отношении к деятельности опыт, по-видимому, ничем не отличается от искусства; напротив, мы видим, что люди» действующие на основании опыта, достигают даже большего успеха, нежели те, которые владеют общим понятием, но не имеют опыта. Дело в том, что опыт есть знание индивидуальных вещей, а искусство — знание общего, между тем при всяком действии и всяком возникновении дело идет об индивидуальной вещи: ведь врачующий излечивает не человека, разве лишь привходящим $^3$(«случайным») образом, а Каллия или Сократа, или кого-либо другого из тех, кто носит это название, — у кого есть привходящее свойство быть человеком.
\alignedmarginpar{$^4$ Аристотель дословно говорит о «техниках» (τους τεχνίτας), имея при этом в виду людей, подобных тем, которые у нас  получили название «ученых специалистов». В соответствии с тем о «мудрости», которая им приписывается, он говорит следующим образом (Elh. Nic. VΙ 7, 1141 а 9—12): «Мудрость в искусствах мы приписываем тем, кто наиболее точно владеет данным искусством, — например, называем мудрым обделывателя камней Фидия и ваятеля Поликлета, разумея  при этом под мудростью не что иное, как то, что она — доброкачественность (αρετή) искусства».}
\alignedmarginpar{$^5$ У Аристотеля здесь дословно противопоставляется «что» (τό δτι) и «почему» (διότι) --- факт и причина факта.}
\alignedmarginpar{$^6$ В скобках [] заключены слова, которых нет в тексте  более ранних рукописей и которые, по-видимому, отсутствовали у Аристотеля.}
Если кто поэтому владеет понятием, а опыта не имеет и общее познает, а заключенного в нем индивидуального не ведает, такой человек часто ошибется в лечении; ибо лечить приходится индивидуальное. Но все же знание и понимание мы приписываем скорее искусству, чем опыту, и ставим людей искусства $^4$ выше по мудрости, чем людей опыта, ибо мудрости у каждого имеется больше, в зависимости от знания: дело в том, что одни знают причину, а другие — нет. В самом деле, люди опыта знают фактическое положение $^5$ (что дело обстоит так-то), а почему так — не знают; между тем люди искусства знают «почему» и постигают причину. Поэтому и руководителям в каждом деле мы отдаем больший почет, считая, что они больше знают, чем простые ремесленники, и мудрее их, так как они знают причины того, что создается. [А с ремесленниками (обстоит дело) подобно тому, как и некоторые неодушевленные существа хоть и делают то или другое, но делают это, сами того но зная (например огонь — жжет): неодушевленные существа в каждом таком случае действуют по своим природным свойствам, а ремесленники — по привычке] $^6$. Таким образом, люди оказываются более мудрыми не благодаря умению действовать, а потому, что они владеют понятием и знают причины. Вообще признаком человека знающего является способность обучать, а потому мы считаем, что искусство является в большей мере наукой, нежели опыт: в первом случае люди способны обучать, а во втором — не способны.

Кроме того, ни одно из чувственных восприятий мы не считаем мудростью, а между тем такие восприятия составляют самые главные наши знания об индивидуальных вещах; но они не отвечают ни для одной вещи на вопрос «почему», например, почему огонь горяч, а указывают только, что он горяч.

Естественно поэтому, что тот, который первоначально изобрел какое бы то ни было искусство за пределами обычных (показаний) чувств, вызвал удивление со стороны людей не только благодаря полезности какого-нибудь своего изобретения, но как человек мудрый и выдающийся среди других. Затем, по мере открытия большого числа искусств, с одной стороны, для удовлетворения необходимых потребностей, с другой — для препровождения времени, изобретатели второй группы всегда признавались более мудрыми, нежели изобретатели первой, так как их науки были предназначены не для практического применения. Когда же все такие искусства были установлены, тогда уже были найдены те из наук, которые не служат ни для удовольствия, ни для необходимых потребностей, и прежде всего (появились они) в тех местах, где люди имели досуг. Поэтому математические искусства образовались прежде всего в области Египта, ибо там было предоставлено классу жрецов время для досуга.

\alignedmarginpar{$^7$ Никомахова этика VI, главы 3—7.}
В этических рассуждениях $^7$ (уже) было указано, в чем разница между искусством, наукой и другими однородными областями; а о чем мы сейчас ведем речь, так это о том, что так называемая мудрость по всеобщему мнению имеет своим предметом первые начала и причины. Поэтому, как уже было сказано ранее, человек, располагающий опытом, оказывается мудрее тех, у кого есть любое чувственное восприятие, а человек, сведущий в искусстве, мудрее тех, кто владеет опытом, руководитель мудрее ремесленника, а умозрительные (теоретические) дисциплины выше созидающих. Что мудрость, таким образом, есть наука о некоторых причинах и началах, это ясно.




\newpage
\section{Книга 2}

\epigraph{
а) Согласно обычным взглядам на мудрого, это --- человек, который в известном смысле знает обо всем; далее --- тот, кто может постигать веши трудные, и тот, кто владеет знанием более точным, а также более способен свое знание передавать; равным образом из наук большее право называться мудростью имеет та, которая представляется ценною сама по себе, и та, которая играет более руководящую роль: всем этим требованиям в наибольшей мере удовлетворяет одна паука --- наука о первых началах и причинах
б) Мудрость (высшая наука) имеет не действенный, но теоретический характер; это явствует и из того, что источником, откуда она появилась, было удавление, и из того, что люди пришли к ней тогда, когда у них уже было все необходимое для удовлетворения жизненных и культурных потребностей.
в) Мудрость по справедливости можно называть божественной и потому, что она в первую очередь подобает богу, и благодаря природе познаваемого ею предмета
г) Исходя от удивления, мудрость в конечном счете приходпт к такому удивлению, которое противоположно первоначальному.
}{Парафраз А.В. Кубицкого}

\alignedmarginpar{$^1$ см. 981 Ь 28—«так называемая мудрость».}
Так как эта именно наука является целью наших поисков, то нужно рассмотреть, какими именно причинами и какими началами должна заниматься наука, которая носит название $^1$ мудрости. Если учесть те мнения, которые у нас есть о мудром человеке, то отсюда вопрос, может быть, выяснился бы больше. Мы предполагаем прежде всего, что мудрый знает все, насколько это возможно, не имея знания в отдельности о каждом предмете. Далее, мы считаем мудрым того, кто в состоянии узнать вещи трудные и не легко постижимые для человека (ведь чувственное восприятие общо всем, а потому это — вещь легкая, и мудрости (в нем) нет никакой). Кроме того, более мудрым во всякой науке является человек более точный и более способный научать, а из наук в большей мере считается мудростью та, которая избирается ради нее самой и в целях познания, а не та, которая привлекает из-за ее последствий и та, которая играет более руководящую роль — в большей мере, чем та, которая занимает служебное положение; ибо мудрому не надлежит получать распоряжения, но давать их, и не он должен повиноваться другому, а ему — тот, кто менее мудр.

\alignedmarginpar{$^2$ А именно — потенциально, в возможности (Ross).}
\alignedmarginpar{$^3$  Или: «которая охватывается этим знанием». Дословно: «все, что подлежит (подведомственно) <этому знанию >».}
\alignedmarginpar{$^4$ Противоречие между утверждением Аристотеля, что  первые начала наиболее трудны для познания, и утверждением, что они наиболее познаваемы, снимается тем, что эти начала, согласно обычному аристотелевскому различению, представляют наибольшие затруднения для  нас, с нашей обычной точки зрения, так как для нас всего ближе непосредственная очевидность чувственного восприятия, но они лучше всего постигаются нашею мыслью, так как все более конкретное наше  знание покоится на достоверном познании этих первопринципов нашим  разумом.}
Такие мнения и столько их имеем мы о мудрости и мудрых людях. Из перечисленных свойств знание обо всем должно быть у того, кто в наибольшей мере владеет знанием в общей форме: такому человеку некоторым образом $^2$ известна вся совокупность вещей (которая входит в круг этого знания). $^3$ Можно сказать, что и наиболее трудны для человеческого познания такие начала — начала наиболее общие: они дальше всего от чувственных восприятий. А наиболее точными являются те из наук, которые больше всего имеют дело с первыми началами: те, которые исходят от меньшего числа (элементов), более точны, нежели те, которые получаются в результате прибавления (новых свойств), например арифметика точнее геометрии. Но и обучать более пригодна та наука, которая рассматривает причины; ибо научают те люди, которые указывают причины для каждой вещи. А знание и понимание, происходящие ради них самих, более всего свойственны науке о предмете, познаваемом в наибольшей мере: $^4$ тот, кто отдает предпочтение знанию ради знания, больше всего отдаст предпочтения науке
\alignedmarginpar{$^5$ Дословно: «науке в наибольшей степени»; «науке по преимуществу» (Роз. и Перв.). Аристотель хочет сказать: науке, которая является таковой в наибольшей степени, т.е. в наибольшей мере имеет характер науки.}
\alignedmarginpar{$^6$ В этом месте термин «подлежащее» (ύποχεί-μενον), в первую очередь означающий у Аристотеля «то, что лежит в основе», «субстрат», может быть принят в своем дословном значении — то, что лежит внизу чего-нибудь, т.е. то, что зависит от чего-нибудь или подчинено чему-нибудь.}
наиболее совершенной, $^5$ а это—наука о максимально познаваемом предмете. Обладают же такою познаваемостью первые элементы и причины, ибо с помощью их и на их основе познается все остальное, а не они через то, что лежит под ними. $^6$ И наиболее руководящей из всех наук, и в большей мере руководящей, чем (всякая) наука служебная, является та, которая познает, ради чего надлежит делать каждую вещь; а такою конечною целью в каждом случае является благо и вообще наилучшее во всей природе.

Все указанные признаки требуют отнести обсуждаемое название к одной и той же науке: это должна быть наука, занимающаяся рассмотрением первых начал и причин; ведь и благо и «то, ради чего» $^7$ также является одною из причин. А что это наука не действенная, ясно и (если судить) по людям, которые первые занялись философией. Ибо вследствие удивления люди и теперь и впервые начали философствовать,
причем вначале они испытали изумление по поводу тех затруднительных вещей, которые были непосредственно перед ними, а затем понемногу продвинулись на этом пути дальше и осознали трудности в более крупных вопросах, например относительно изменений $^8$ луны и тех, которые касаются солнца и звезд, а также — относительно возникновения мира. $^9$

\alignedmarginpar{$^7$ Выражение «то, ради чего» Аристотель употребляет как технический термин для обозначения цели.}
\alignedmarginpar{$^8$ παθήματα — «испытываемые состояния». В первую очередь здесь, вероятно, имеются в виду фазы луны и затмения солнца.}
\alignedmarginpar{$^9$ Дословно: «всего» (τοο παντός), но не в смысле всех вещей, а в смысле всеобъемлющего целого.}
Но тот, кто испытывает недоумение и изумление, считает себя незнающим (поэтому и человек, который любит мифы, является до некоторой степени философом, ибо миф слагается из вещей, вызывающих удивление). Если таким образом начали философствовать, убегая от незнания, то, очевидно, к знанию стали стремиться ради постижения (вещей), а не для какого-либо пользования (ими). То, что произошло на деле, подтверждает это: когда оказалось налицо почти все необходимое и также то, что служит для облегчения жизни и препровождения времени, тогда (только) стало предметом поисков такого рода разумное мышление. Ясно поэтому, что мы не ищем его ни для какой другой нужды. Но как свободный человек, говорим мы, это — тот, который существует ради себя, а не ради другого, так ищем мы и эту науку, так как она одна только свободна изо (всех) наук: она одна существует ради самой себя.

\alignedmarginpar{$^{10}$ Т.е. находится в зависимости от внешней обстановки.}
\alignedmarginpar{$^{11}$ Симонид — греческий поэт, лирик, родом с острова Ксоса (2-я половина VI века — начало V века до нашей эры).}
Поэтому и достижение ее по справедливости можно бы считать не человеческим делом; ибо во многих отношениях являет природа людей рабские черты $^{10}$, так что, по словам Симонида $^{11}$, «бог один иметь лишь мог бы этот дар», человеку же подобает искать соразмерного ему знания. Если поэтому слова поэтов чего-нибудь стоят, и божеской природе свойственна зависть, естественнее всего ей проявляться в этом случае, и несчастны должны бы быть все, кто хочет слишком многого. Но не может этого быть, чтобы божественное существо было проникнуто завистью, — напротив, и по пословице «лгут много поэты», — и не следует (какую-либо) другую науку считать более ценною, чем эту. В самом деле, наука наиболее божественная является и наиболее ценною. А божественною может считаться только одна эта — с двух точек зрения. На самом деле, божественною является та из наук, которою скорее всего мог бы владеть бог, и точно так же — если есть какая-нибудь наука о божественных предметах. А только к одной нашей науке подходит и то и другое. Бог по всеобщему мнению находится в числе причин и есть одно из начал, и такая наука могла бы быть или только у одного бога, или у бога в наибольшей мере. Таким образом, все науки более необходимы, нежели она, но лучше — нет ни одной.

\alignedmarginpar{$^{12}$ Слова «кто еще не рассмотрел причину» в тексте рукописей стоят после слов: «про загад. самодвиж. игрушки». Ross в своем издании переносит их на то место, где они стоят в нашем переводе, ссылаясь на авторитет Jager'a и Bonitz'a (Bon., правда, сам этой перестановки сделать не решился, — Comm. 57).}
Вместе с тем овладение этой наукой должно у нас известным образом привести к противоположным результатам по сравнению с первоначальными исканиями. Как мы говорили, все начинают с изумления, обстоит ли дело именно так: как (недоумевают), например, про загадочные самодвижущиеся игрушки, или (сходным образом) в отношении солнцеворотов, или несоизмеримости диагонали; ибо у всех, (кто еще не рассмотрел причину) $^{12}$, вызывает удивление, если чего-нибудь нельзя измерить самою малою мерою. А под конец нужно прийти к противоположному — и к лучшему, как говорится в пословице, — этим дело кончается и в приведенных случаях, когда в них разберутся: ведь ничему бы так не удивился человек сведущий в геометрии, чем если бы диагональ оказалась измеримой.

Теперь сказано, какова природа искомой науки и какова цель, к которой должно прийти искание (этой науки) и все вообще исследование.





\newpage
\section{Книга 3}

\epigraph{
Имеется в общем четыре рода основных начал, --- это было показано в физике и подтверждается авторитетом древних философов, которые сверх этих родов не могли больше найти ни одного. Самые древние философы привимали только материальную причину вещей --- воду, воздух или другие простейшие тела; затем, побуждаемые фактическим положением дел, они стали к материи присоединять причину движения, --- за исключением тех из них, которые пришли к убеждению, что вся совокупность вещей неподвижна. После этого Анаксагор, также под влиянием действительности, впервые понял необходимость установить причину третьего рода, из которой можво было бы вывести все, что есть хорошего во всей природе, однако же он не отделил эту причину от причины движущей.
}{Парафраз А.В. Кубицкого}

\alignedmarginpar{$^1$ Аристотель часто заменяет положительную формулировку понятия неопределенным обозначением его как ответа на тот или иной вопрос. Так, например, — «то, откуда начало движения» οθενή αρχή τής κινήσεως, — источник движения; «то, ради чего» (τό ού ενεκα), — цель. Подобный же характер носит его знаменитая формула для обозначения логической сущности: "что есть (или—что представляет собою, дословно — чем было) бытие (для такой-то вещи) (τό τί ήν είναι)" — сущность бытия вещи. Я перевожу эту формулу техническим выражением «суть бытия».}
\alignedmarginpar{$^2$ Как отмечает Ross,— «взятое в конечном счете основание, почему (вещь такова, как она есть)» и «то основное, благодаря чему (вещь именно такова)», весьма неуклюжим образом указывают на одно и то же («формальное») начало.}
\alignedmarginpar{$^3$ Точнее — начала, относящиеся к разряду материи. Eidos, о котором здесь говорится, не имеет в  данном  случае, разумеется, обычного у Аристотеля технического значения — «формы», но должно переводиться как «вид» (логический) или «разряд», «группа»: началами всех вещей у первых философов являются лишь те, которые принадлежат к виду или разряду материи (т.е. к группе начал материальных).}
\alignedmarginpar{$^4$ У Аристотеля — «музический» (μουσικός), — постоянный пример для качества. В этом употреблении, может быть, точнее всего будет передача «образованный», в противоположность αμουσος — необразованный, грубый; для «музыки» (μουσική), вернее «музического искусства», у Платона — обширный ряд оттенков от поэзии до философии (Phaed. 60 D-61 А).}
\alignedmarginpar{$^5$ Так передает здесь φύσις (дословно — «природа») Маковельский; по смыслу можно перевести «от природы существующее бытие».}
Нам очевидным образом надлежит достигнуть знания о первоначальных причинах, — мы ведь тогда приписываем себе знание каждой вещи, когда, по нашему мнению, мы постигаем первую причину. А о причинах речь может идти в четырех смыслах: одной такой причиной мы признаем сущность и суть бытия («основание, почему» $^1$ (вещь такова, как она есть), восходит в конечном счете к понятию вещи, а то основное, благодаря чему (вещь именно такова), есть некоторая причина и начало); $^2$ другой причиной мы считаем материю и лежащий в основе субстрат; третьей — то, откуда идет начало движения; четвертой — причину, противолежащую (только что) названной, а именно — «то, ради чего» (существует вещь), и благо (ибо благо есть цель всего возникновения и движения). Вопрос об этих причинах был, правда, в достаточной степени рассмотрен у нас в книгах о природе, но все же привлечем также и тех, которые раньше нас обратились к исследованию вещей и вели философские рассуждения об истинном бытии. Ибо очевидно, что и они указывают некоторые начала и причины. Поэтому если мы переберем их учения, (от этого) будет некоторая польза для теперешнего исследования: или мы найдем какой-нибудь другой род причин, или больше будем верить тем, которые указываются в настоящее время.

Из тех, кто первые занялись философией, большинство считало началом всех вещей одни лишь начала в виде $^3$ материи: то, из чего состоят все вещи, из чего первого они возникают и во что в конечном счете разрушаются, причем основное существо пребывает, а по свойствам своим меняется, — это они считают элементом и это — началом вещей. И вследствие этого они полагают, что ничто не возникает и не погибает, так как подобная основная природа всегда сохраняется, подобно тому, как и про Сократа мы не говорим ни — что он становится просто, когда он становится прекрасным или образованным $^4$ , ни — что он погибает, когда он утрачивает эти свойства, ввиду того, что пребывает лежащий в основе субстрат — сам Сократ. Таким же образом (не допускают они возникновения и погибели) и для всего остального; ибо должно быть некоторое природное естество $^5$ — или одно, или больше, чем одно, откуда возникает все остальное, причем само это естество остается в сохранности.

\alignedmarginpar{$^6$ Дословно: «отцами».}
\alignedmarginpar{$^7$ Гомер говорит: «Океана источник богов и мать их Тефиду» (Илиада, XV, 201, 246), причем, однако, под богами надлежит здесь разуметь не богов вообще, коих Океан и Тефида родителями не были, а женские морские божества — Океанид. Кого Аристотель имеет здесь в виду, кроме Гомера, установить трудно. По мнению Alex., речь может еще идти о Гесиоде. Есть также некоторые основания относить это указание Аристотеля к космогониям так называемых орфиков, возводивших свои учения к мифическому певцу Орфею.}
\alignedmarginpar{$^8$ О Гиппоне, жившем в эпоху Перикла, Аристотель дает резко отрицательный отзыв еще и в другом месте (de anima 1, 2, 405, b 2). Необходимо, однако, отметить, что Гиппон был убежденным материалистом, не признававшим никакого бытия помимо чувственно воспринимаемых тел, вследствие чего он и получил прозвание безбожника (Alex. Comm. 428, 21—23). Отсюда, может быть, и антипатия Аристотеля.}
\alignedmarginpar{$^9$ О возникновении и уничтожении элементов по Эмпедоклу можно говорить лишь в том смысле, что отдельные из них накапливаются иногда в большом, иногда в малом количестве на, пути к объединению по отдельным элементам (под влиянием действия вражды); и затем рассеиваются из этого объединения (см. Arlst Met. А 4,985 а 25—28 и Ross- I 181).}
\alignedmarginpar{$^{10}$ Здесь говорится о принимавшихся Анаксагором элементарных предметах с однородными друг другу частями (у Анаксагора они назывались семенами вещей, а позже у Аристотеля получили название «гомеомерных», — подобночастных — вещей, или гомеомерий). Аристотель в нашем месте иллюстрирует их примером огня и воды, так как сам он, в отличие от Анаксагора, также видел в этих последних вещи с однородными частями. Для анаксагоровских гомеомерий устанавливается один путь генезиса и уничтожения — через соединение и разделение; ограничение «почти» вводится Аристотелем с своей точки зрения потому, что возникновение через соединение однородного как раз не относится к обычным (эмпедокловским) элементам, которые Анаксагор считал не за элементы, но за смеси из всех первичных «семян» (Ср. Ross I 132—133 и цитируемое им место Arist. De Coelo Г 3, 302 α 28 сл).}
\alignedmarginpar{$^{11}$ Т.е. причину движения.}
Количество и форму для такого начала не указывают все одинаково, но Фалес --- родоначальник такого рода философии --- считает его водою (вследствие чего он и высказывал мнение, что земля находится на воде); к этому предположению он, можно думать, пришел, видя, что пища всех существ --- влажная и что само тепло из влажности получается и ею живет (а то, из чего (все) возникает, это и есть начало всего). Таким образом он отсюда пришел к своему предположению, а также потому, что семена всего (что есть) имеют влажную природу, а у влажных вещей началом их природы является вода.

Есть и такие, которые полагают, что и (мыслители) очень древние, жившие задолго до теперешнего поколения и впервые занявшиеся теологией, держались именно таких взглядов относительно природы: Океана и Тефиду они сделали источниками $^6$
возникновения $^7$, и клятвою богов стала у них вода, а именно Стикс, как они его называли; ибо почтеннее всего — самое старое, а клятва, это — самое почтенное. Во всяком случае, является ли это мнение о природе древним и давнишним, это, может быть, и недостоверно, но о Фалесе говорят, что он так высказался относительно первой причины (что касается Гиппона $^8$,его, пожалуй, не всякий согласится поставить рядом с этими философами вследствие ограниченности его мысли). С другой стороны, Анаксимен и Диоген ставят воздух раньше, нежели воду, и из простых тел его главным образом принимают за начало; Гиппас из Метапонта и Гераклит из Эфеса (выдвигают) огонь, Эмпедокл — (известные) четыре элемента, к тем, которые были названы, на четвертом месте присоединяя землю; элементы эти всегда пребывают, и возникновение для них обозначает только (появление их) в большом и в малом числе в то время, когда они собираются (каждый) в одно и рассеиваются из одного. $^9$ А Анаксагор из Клазомен, будучи по возрасту раньше этого последнего, а по делам своим позже его, утверждает, что начала не ограничены (по числу): по его словам, почти все подобночастные предметы, являющиеся таковыми по образцу воды или огня $^{10}$, возникают и уничтожаются именно таким путем — только через соединение и разделение, а иначе не возникают и не уничтожаются, но пребывают вечно.

Исходя из этих данных, за единственную причину можно было бы принять ту, которая указывается в виде материи. Но по мере того как они в этом направлении продвигались вперед, самое положение дела указало им путь и со своей стороны принудило их к (дальнейшему) исследованию. В самом деле, пусть всякое возникновение и уничтожение сколько угодно происходит на основе какого-нибудь одного или хотя бы нескольких начал, почему оно происходит, и что — причина этого? Ведь не сам лежащий в основе субстрат производит перемену в себе, например ни дерево, ни медь (сами) не являются причиной, почему изменяется каждое из них, и не производит дерево — кровать, а медь — статую, но нечто другое составляет причину (происходящего) изменения. А искать эту причину — значит искать другое начало, как мы бы сказали — то, откуда начало движения. Те, которые с самого начала взялись за подобное исследование и утверждали единство лежащего в основе субстрата, не испытывали никакого недовольства собой, но, правда, некоторые из (таких) сторонников единства, как бы под давлением этого исследования, объявляют единое неподвижным, как равно и всю природу, не только в отношении возникновения и уничтожения (это — учение старинное, и все с ним соглашались), но и в отношении всего остального изменения; и это их своеобразная черта. Из тех, таким образом, кто объявлял мировое целое единым, никому не довелось усмотреть указанную (сейчас) причину $^{11}$, разве только Пармениду, да и этому последнему — постольку, поскольку он полагает не только существование единого, но — в известном смысле — и существование двух причин. Тем же, кто вводит множественность (начал), скорее можно говорить (о такой причине), например тем, кто принимает теплое и холодное, или огонь и землю: они пользуются огнем, как обладающим двигательного природой, а водою, землей и тому подобными (элементами) на противоположный лад.

\alignedmarginpar{$^{12}$ αίτίαν του καλώς) — дословно: причину «того, что <вещи находятся> в хорошем состоянии».}
После этих философов и такого рода начал, так как эти последние были недостаточны, чтобы вывести из них природу вещей, стали, снова побуждаемые — как мы сказали — самой истиной, искать следующего затем начала. Что одни вещи находятся в хорошем и прекрасном состоянии, а другие приходят к нему в процессе своего возникновения, причиной этого не подобает быть ни огню, ни земле, ни чему-либо другому в этом роде, и те философы подобного  взгляда наверно и не держались; а с другой стороны, не хорошо было также вверять такое дело случаю и самопроизвольному процессу. Поэтому тот, кто сказал, что разум находится, подобно тому как в живых существах, также и в природе, и что это он — виновник благоустройства мира и всего мирового порядка, этот человек представился словно трезвый по сравнению с пустословием тех, кто выступал раньше. Явным образом, как мы знаем, взялся за такие объяснения Анаксагор, но есть указание, что прежде об этом сказал Гермотим из Клазомен. Те, которые стояли на этой точке зрения, в то же время признали причину совершенства $^{12}$ (в вещах) первоначалом вещей, и притом — таким, от которого вещи имеют движение.






\newpage
\section{Книга 4}

\epigraph{
Сходного с Анаксагором взгляда на движущую причину держались, повидимому, Гесиод и Парменид. Так как в природе порядок сохраняется не всюду, но иногда бывает и нарушен, то Эмпедокл установил двоякую причину движения: одну --- для хорошего, другую --- для дурного. Перечисленные философы устанавливали два начала: материальное и движущее, но Анаксагор движущую причину для объяснения природы использовал мало, а Эмпедокл не остается достаточно оследовательным и, помимо установления двух движущих причин, делает еще то нововведение, что у него впервые появляются четыре материальных элемента (которые при этом могут быть сведены к основным двум). Наконец, Левкипп и Демокрит признают началами вещей полное и пустое и объясняют разнообразие вещей из формы, порядка и положения элементов, а вопроса о движущей причине не ставят.
}{Парафраз А.В. Кубицкого}

\alignedmarginpar{$^1$ «Теогония» Гесиода, стихи 116—120.}
\alignedmarginpar{$^2$ Или — «выставлял первую как источник положительных, вторую — как источник отрицательных свойств.}
Можно предположить, что Гесиод первый стал искать нечто в этом роде, а также — если еще кто принял для вещей любовь или страсть в качестве начала, — как, например, (поступил) и Парменид: ведь и он, устанавливая возникновение вселенной, замечает: «Всех богов первее Эрот был ею замышлен». А по словам Гесиода: «В самую первую очередь хаос возник, а затем уж Гея (земля) с широкою грудью»... «также — Эрот, что меж всех бессмертных богов выдается» $^1$ ; ибо должна быть среди вещей некоторая причина, которая будет приводить в движение и соединять вещи. Как этих людей распределить [в отношении того], кто (из них высказался по этому вопросу) первый, (об этом) да будет позволено иметь суждение позже; а так как в природе явным образом были (на лицо) и вещи, противоположные хорошим, и не только — стройность и красота, но также — нестроение и уродство, причем дурного было больше, чем хорошего, и безобразного — больше, чем прекрасного, поэтому другой мыслитель ввел дружбу и вражду, выставляя каждую из них как источник [каждого] (соответственного) ряда свойств. $^2$ В самом доле, если последовать за Эмпедоклом и взять его (слова) по смыслу, а не по тому, что он лепечет в своих речах, можно будет найти, что дружба есть причина хорошего, а вражда — причина дурного. И потому, если сказать, что в известном смысле Эмпедокл признает — и (притом) первый признает — зло и добро за начала, то это, пожалуй, будет сказано хорошо, поскольку причиною всех благ является (у него) само благо, а причиною зол — зло.

\alignedmarginpar{$^3$ Дословно: «скорее все выставляет причиной происходящих вещей, нежели ум».}
\alignedmarginpar{$^4$ Согласно чтению Ross'а — τό τήν αίτίαν διελεΐν, вместо прежде читавшегося ταύτην τήν αίτίαν διελών (так как в окружающем тексте не к чему относить ταύτην).}
Перечисленные философы, как мы отмечаем, до сих пор очевидным образом привлекли две причины из тех, которые мы различили в книгах о природе — материю и источник движения, при этом нечетко и без всякой ясности, но как это делают в битвах люди неискусные; ведь и те, оборачиваясь во все стороны, наносят иногда прекрасные удары, но не потому, что знают; и точно так же указанные философы не производят впечатление людей, знающих, что они говорят: они явным образом совсем почти не пользуются своими началами или в (очень) малой мере. Анаксагор использует ум как машину для создания мира, и когда у него явится затруднение, в силу какой причины (то или другое) имеет необходимое бытие, тогда он его привлекает, во всех же остальных случаях он все, что угодно, выставляет причиною происходящих вещей, но только не ум. $^3$ А Эмпедокл обращается к причинам больше, нежели Анаксагор, но и он обращается недостаточно и, имея с ними дело, не получает последовательных результатов. По крайней мере у него во многих случаях дружба разделяет, а вражда соединяет. В самом деле, когда целое под действием вражды распадается на элементы, тогда огонь собирается вместе и также — каждый из остальных элементов. Когда же элементы снова под действием дружбы сходятся в единое целое, то из каждого элемента части (его) должны опять рассеяться (в разные стороны).

Эмпедокл, таким образом, в отличие от прежних философов первый ввел разделение (движущей) причины $^4$ — установил не одно начало движения, а два разных, и притом противоположных. Кроме того, элементы, относимые к разряду материи, он первый указал в числе четырех (он однако же не пользуется ими как четырьмя, а точно их только два: у него (на одной стороне) отдельно — огонь, а (на другой) противоположные (огню) — земля, воздух и вода как одно вещество. Это можно было бы вывести, стоя [в своем рассмотрении] на основе его произведений).

\alignedmarginpar{$^5$ Слова «и разреженное» одна из главных рукописей ($A^b$) выпускает, и их есть основание рассматривать как раннюю вставку.}
\alignedmarginpar{$^6$ Ross указывает, что ρυσμός есть правильная ионическая форма вместо ρυθμός, и приводит ряд параллельных мест, где ρυθμός употреблено в смысле «форма» (в том числе у Геродота и Ксенофонта).}
\alignedmarginpar{$^7$ В примере, приводимом Аристотелем, греческие буквы указываются так: «А отличается от N формой, ΑΝ от ΝΑ порядком, Η (современный Аристотелю знак для Ζ) от Η положением» (Ross, 1,141).}
Эмпедокл, как мы указываем, установил такие начала и столько их; а Левкипп и его сотоварищ Демокрит признают элементами полное и пустое, называя одно сущим, другое небытием, а именно: полное и твердое — сущим, а пустое [и разреженное] $^5$ — небытием (поэтому они и говорят, что бытие существует отнюдь не более, чем небытие, потому что и тело — не больше, чем пустота), причиною же вещей является то и другое как материя. И как мыслители, утверждающие единство основной субстанции, все остальное выводят из ее состояний, принимая разреженное и плотное за начала (всех таких) состояний, таким же образом и эти философы считают основные отличия (атомов) причинами всех других свойств. А этих отличий они указывают три: форму, порядок и положение. Ибо бытие по их словам различается лишь «строем» $^6$, «соприкосновением» и «поворотом»; в том числе «строй», это — форма, «соприкосновение» — порядок, «поворот» — положение; в самом деле, "А" отличается от "Р" формой, "АР" от "РА" порядком, "Ь" от "Р" положением. $^7$ А вопрос о движении, откуда или в какой форме получится оно у вещей, эти мыслители также, подобно остальным, беспечно оставили в стороне. Таким образом, относительно двух причин исследование, по-видимому, произведено прежними философами в указываемых пределах.






\newpage
\section{Книга 5}

\epigraph{
Пифагорейцы полагали, что природа вещей усматривается в числах, а элементами чисел объявили чет и нечет (из коих первый знаменует в составе числа неопределенную, второй --- определенную природу). Некоторые пифагорейцы признавали началами десять пар противоположностей; подобно им считал началами пары против·>положностей и Алкмеон Кротонский, не указывая однако определ иного числа этих пар. Устанавливая такие начала, пифагорейцы, повидимому, относили их к группе причин материальных. Что касается элейпев, признававших единое неизменное бытие, то их надлежит принять здесь во внимание лишь в той мере, поскольку одни из лих приписывали этому быитю логический, другие --- материальный характер, а также --- поскольку Пармепид пытался дать объяснение миру человеческого «мнения». Подводится итог всем дошедшим до нас взглядам древних философов на природу объясняющих мир начал и отмечается, что пифагорейцы, попидимому, впервые стали исследова ь логическую (формальную) причину, но отнеслись к этому делу очень поверхностно
}{Парафраз А.В. Кубицкого}

\alignedmarginpar{$^1$ Дословно: «много сходств».}
\alignedmarginpar{$^2$ Т.е. выражение свойств и отношений, присущих гармоническим сочетаниям.}
\alignedmarginpar{$^3$ Pragmateia = и усиленная работа и получающееся в результате произведение.}
\alignedmarginpar{$^4$ Вымышленное тело, которое пифагорейцы помещали между землею и также ими вымышленным центральным огнем. С земли противоземля не видна, как не видно и центрального огня, потому что люди находятся только на внешней части земли, никогда не поворачивающейся к мировому центру.}
\alignedmarginpar{$^5$ По указанию Alex. (Comm.), имеются в виду сочинение «О небе» (II 13) и утраченная работа "Взгляды пифагорейцев" (αί των Πυθαγορικών δόςα).}
\alignedmarginpar{$^6$ По объяснению Ross'a Аристотель хочет сказать, что пифагорейцы приписывали числу характер материальной причины и в то же время делали первые попытки понять его в качестве причины формальной, так что έςεις есть достаточно естественный эквивалент для εΐδη (Ross I, 147.)}
Одновременно с этими философами и (еще) раньше их так называемые пифагорейцы, занявшись математическими науками, впервые двинули их вперед и, воспитавшись на них, стали считать их начала началами всех вещей. Но в области этих наук числа занимают от природы первое место, а у чисел они усматривали, казалось им, много сходных черт $^1$ с тем, что существует и происходит, — больше, чем у огня, земли и воды, например такое-то свойство чисел есть справедливость, а такое-то душа и ум, другое — удача, и можно сказать — в каждом из остальных случаев точно так же. Кроме того, они видели в числах свойства и отношения $^2$, присущие гармоническим сочетаниям. Так как, следовательно, все остальное явным образом уподоблялось числам по всему своему существу, а числа занимали первое место во всей природе, элементы чисел они предположили элементами всех вещей и всю вселенную (признали) гармонией и числом. И все, что они могли в числах и гармонических сочетаниях показать согласующегося с состояниями и частями мира и со всем мировым устройством, это они сводили вместе и приспособляли (одно к другому); и если у них где-нибудь того или иного не хватало, они стремились (добавить это так), чтобы все построение $^3$ находилось у них в сплошной связи. Так, например, ввиду того, что десятка (декада), как им представляется, есть нечто совершенное и вместила в себе всю природу чисел, то и несущихся по небу тел они считают десять, а так как видимых тел только девять, поэтому на десятом месте они помещают противоземлю. $^4$ В другом сочинении у нас об этом разъяснено точнее. $^5$ А ради чего мы касаемся этой области, так это — для того, чтобы и у них установить, какие они полагают начала и как начала эти сводятся к указанным выше причинам. Во всяком случае и у них, по-видимому, число принимается за начало и в качестве материи для вещей и в качестве (выражения для) их состояний и свойств $^6$, а элементами числа они считают чет и нечет, из коих первый является неопределенным, а второй определенным; единое состоит у них из того и другого, — оно является и четным и нечетным, число (образуется) из единого, а (различные) числа, как было сказано, это — вся вселенная.

\alignedmarginpar{$^7$ Последнее противопоставление обычно толкуется как противопоставление равностороннего четыреугольника (специально — квадрата) разностороннему прямоугольнику.}
\alignedmarginpar{$^8$ Bonitz удачно переводит: «... был младшим современником Пифагора»... Может быть, дословно можно было бы передать: «... пришелся на старые годы II.». Diels предлагает — νέος έπι γέρ. Πυθ: «был молодым, когда Пифагор был стариком».}
\alignedmarginpar{$^9$ Мы бы сказали: имманентных}
Другие из этих же мыслителей принимают десять начал, идущих (каждый раз) в одном ряду — предел и беспредельное, нечет и чет, единое и множество, правое и левое, мужское и женское, покоящееся и движущееся, прямое и кривое, свет и тьму, хорошее и дурное, четыреугольное и разностороннее. $^7$ Такого же мнения, по-видимому, держался и Алкмеон Кротонец, и либо он перенял это учение у тех мыслителей, либо они у него. И по времени ведь Алкмеон приходится на годы старости Пифагора $^8$, а высказался он подобно им. Он утверждает, что большинство свойств, с которыми имеют дело люди, составляют пары, но указывает противоположности — не определенные, как те мыслители, а какие случится, например белое — черное, сладкое — горькое, хорошее — дурное, большое — малое. Он, таким образом, отбросил остальные вопросы, не давая (ближайших) определений, а пифагорейцы указали и сколько противоположностей и какие они. И в том и в другом случае мы, следовательно, узнаем, что противоположности суть начала вещей; но сколько их — узнаем у одних пифагорейцев, и также — какие они. А как можно (принимаемые пифагорейцами начала) свести к указанным выше причинам, это у них ясно не расчленено, но, по-видимому, они помещают свои элементы в разряд материи; ибо по их словам из этих элементов, как из внутри находящихся $^9$ частей, составлена и образована сущность.


\alignedmarginpar{$^{10}$ Или: «которые высказались о совокупности бытия, что это — единая природа...»}
\alignedmarginpar{$^{11}$ Аристотель хочет указать, что теории, созданные относительно природы сущего различными философами Элейской школы, различались между собой как в отношении связности (и, следовательно, убедительности) аргументации, так и относительно характера, приписываемого ими истинному бытию (Bon. Comm. 83).}
\alignedmarginpar{$^{12}$ Дословно: «речь о них не стоит никоим образом в связи с теперешним рассмотрением причин».}
\alignedmarginpar{$^{13}$ Аристотель говорит о «физиологах», употребляя это слово в непосредственном смысле и называя так мыслителей, в общей форме рассуждавших о природе («физиология» — философия природы, натурфилософия).}
\alignedmarginpar{$^{14}$ Т.е. ни логической, ни материальной.}
\alignedmarginpar{$^{15}$ У Аристотеля — «у них». Может быть «у этих <двух> мыслителей», как принято в русском переводе. Ross объясняет: «у этих начал», хотя здесь как раз трудно говорить о началах, потому что в Элейской философии по Аристотелю бытие именно не выступает как «причина».}
На основании сказанного (можно) достаточно познакомиться с образом мыслей древних философов, устанавливавших для природы несколько элементов. Есть, однако, и такие, которые высказались о вселенной как об единой природе $^{10}$, но при этом высказались не одинаково — ни в отношении удачности сказанного, ни в отношении существа дела. $^{11}$ Правда, в связи с теперешним рассмотрением причин говорить о них  совсем не приходится $^{12}$ (они не поступают подобно некоторым натурфилософам $^{13}$, которые, предположив сущее единым, вместе с тем производят (вселенную) из единого, как из материи, но высказываются иначе, — те мыслители добавляют (к единому) движение, по крайней мере когда производят вселенную, а эти оставляют его неподвижным). Но вот что во всяком случае имеет отношение к теперешнему исследованию. Парменид, по-видимому, занимается единым, которое соответствует понятию, Мелисс — единым, которое соответствует материи. Поэтому один объявляет его ограниченным, другой — неограниченным; а Ксенофан, который раньше их всех принял единство (говорят, что Парменид был его учеником), ничего не различил ясно и не коснулся ни той, ни другой природы $^{14}$, (указанной этими мыслителями) $^{15}$, но, воззревши на небо в его целости, он заявляет, что единое, вот что такое бог. Этих мыслителей, как мы сказали, с точки зрения теперешнего исследования, надлежит оставить в стороне, двоих притом, именно Ксенофана и Мелисса, даже совсем — так как они немного грубоваты; что же касается Парменида, то в его словах, по-видимому, больше проницательности. Признавая, что небытие отдельно от сущего есть ничто, он считает, что по необходимости существует (только), одно, а именно — сущее, и больше ничего (об этом мы яснее сказали в книгах о природе).

\newpage

\alignedmarginpar{$^{16}$ В греческом тексте здесь (а 3 — 8) фактически — значительно более сложная конструкция, характер которой в общих чертах можно бы было передать следующим образом: «<получили мы это> от наиболее ранних из них, которые полагали, что начало является телесным..., причем некоторые принимали, что телесное начало одно, а другие, что таких начал несколько, но те и другие отнесли их в разряд материи; а также — от некоторых таких философов, которые принимали и эту причину, и кроме нее — ту...» (см. Ross I 155 к строкам а 4—9).}
\alignedmarginpar{$^{17}$ Аристотель не хочет сказать, что пифагорейцы признали те же два начала, как «физиологи», принимавшие материю и причину движения, а только, что они также принимали в общем два начала (здесь это будут — материя и принцип формы).}
\alignedmarginpar{$^{18}$ Т.е. не составляют лишь свойства или определения отличных от них по существу стихий.}
\alignedmarginpar{$^{19}$ Т.е. чему даются такие определения (что определяется как сущее и единое).}
\alignedmarginpar{$^{20}$ Относительно «того, что есть вещь» (περι του τί εστίν), — здесь разумеется ответ на вопрос о существе вещи, ср. выше примечание 1 к 3-й главе.}
\alignedmarginpar{$^{21}$ По мнению Ross'a (см. I, 156—157), Аристотель отмечает «упрощенность» использования формальной причины у пифагорейцев в двух отношениях. С одной стороны, даваемые ими определения были поверхностны, они выхватывали какие-нибудь внешние признаки определяемого предмета, например (Alex. Comm. 36,26), существо дружбы видели в воздаянии равным за равное, или существо справедливости — в испытании равного содеянному, хотя это только — отдельные моменты в понятии дружбы или справедливости. С другой стороны, принятые ими определения они сближали с первым обладавшим аналогичными свойствами числом, хотя этими свойствами обладал и ряд последующих чисел, а потому между их определениями и указывавшимися для выражения сущности определяемого предмета числами тоже не было полного соответствия (обратимости). Так, например, произведение равного на равное (ίσάχις ϊσον), по их мнению, находило себе выражение в числе 4, а потому они видели в числе 4 сущность всего, что характеризовалось как сочетание равного с равным (дружба, справедливость и т.д.): получались такие же последствия, как если бы мы стали считать, что «двойное» исчерпывается числом два, между тем как на самом деле два есть лишь первый случай «двойного», и двойное применимо также к четырем, к шести и т.д.; при таком ходе рассуждения, к одному и тому же — к двум — должны бы были у пифагорейцев свестись и эти дальнейшие числа, и все те вещи, к которым применимо отождествленное с числом два определение «двойное».}
\alignedmarginpar{$^{22}$ Ross поясняет — «более поздних в доплатоновском периоде».}
Однако же вынуждаемый сообразоваться с явлениями и признавая, что единое существует соответственно понятию, а множественность — соответственно чувственному восприятию, он затем устанавливает две причины и два начала — теплое и холодное, а именно говорит об огне и земле; причем из этих двух он к бытию относит теплое, а другое начало — к небытию.

Вот все, что мы извлекли из сказанного ранее и у мудрецов, принявших уже участие в выяснении этого вопроса. Первые, из них признавали $^{16}$, что начало является телесным (вода, огонь и току подобные вещи суть тела), причем некоторые принимали, что телесное начало одно, а другие — что таких начал несколько, но те и другие относили их в разряд материи; а были такие, которые принимали и эту причину и кроме нее — ту, от которой исходит движение, причем эту последнюю некоторые (признавали) одну, а другие (указывали) (их) две.

Вплоть до италийских философов и не считая их, все остальные высказались о началах в довольно скромной мере, но только, как мы сказали, воспользовались двумя причинами, и из них вторую — ту, откуда движение — некоторые устанавливают одну, а другие (вводят их) две. Что же касается пифагорейцев, то они таким же точно образом сказали, что есть два начала $^{17}$, а прибавилось у них, как раз и составляя их отличительную черту, только то, что ограниченное, неограниченное и единое — это, по их мнению, не свойства некоторых других физических реальностей $^{18}$, например огня или земли, или еще чего-нибудь в этом роде, но само неопределенное и само единое были (у них) сущностью того, о чем (то и другое) сказываются $^{19}$, вследствие чего число и составляло (у них) сущность всех вещей. Вот каким образом высказались они по этому вопросу, и относительно сути (вещи) $^{20}$ они начали рассуждать и давать (ее) определение, но действовали слишком упрощенно. Определения их были поверхностны $^{21}$, и то, к чему указанное (ими) определение подходило в первую очередь, это они и считали сущностью предмета, как если бы кто думал, что двойное и число два есть одно и то же, потому что двойное составляет в первую очередь свойство двух. Но ведь пожалуй что не одно и тоже — быть двойным и быть двумя, а в противном случае одно (и то же) будет несколькими (разными) вещами, как это получалось также и у них. — Таким образом, вот все, что можно извлечь у более ранних философов и у других $^{22}$ (которые были после них).






\newpage
\section{Книга 6}

\epigraph{
Под влиянием гераклитовского учения Платон признал невозможным познание чувственных вещей. Обратившись затем по примеру Сократа к исследованию общих понятий, он установил особые реальности --- идеи, отличные от чувственных вещей, а эти последние объявил существующими «через приобщение» к идеям. Кроме чувственных вещей и идей, он --- посредине между теми и другими --- поместил математические вещи и элементы идей признал элементами всех вещей. Указывается, что есть в учениях Платона общего с пифагорейцами и чт5 отлично от них, а также --- какими он пользовался родами причин.
}{Парафраз А.В. Кубицкого}

\alignedmarginpar{$^1$ В отличие от Воn. я связываю вместе не των όντων ιδέας, а τά τοιαύτα των όντων}
\alignedmarginpar{$^2$ Ross (I 161) высказывается против того, чтобы здесь подразумевать «существуют», и ставит παρά ταύτα в зависимость от λέγεσαι, причем παρά здесь толкует аналогично с его значением в слове παρώνυμος: «а чувственные вещи обозначаются в зависимости от них и сообразно с ними». Но тогда разница в смысле между παρά ταύτα и ατά ταύτα оказывается слишком незначительной.}
\alignedmarginpar{$^3$ Следуя Gillesple и Ross'у (I 161—2), я исключаю τοις εΐδεσιν и ставлю των συνωνύμων (под «одноименными сущностями» разумеются однородные чувственным вещам и по имени и по сущности идеи) в зависимость не от τά πολλά, а от χατά μέθεξιν.}
\alignedmarginpar{$^4$ Слова «изменивши имя» Christ вслед за одною из рукописей ($A^b$) выпускает из текста, как ненужное повторение только что сказанного, попавшее в текст впоследствии.}
\alignedmarginpar{$^5$ Если не считать вместе с Christ'ом «числа», а вместе с Gillespie и Целлером «идеи» позднейшей вставкой, то надо видеть в числах пояснение к «идеям». Это дает вполне приемлемый смысл, ибо «единое», с одной стороны, «большое и малое» — с другой, выступили у Платона как основные начала в тот период, когда идеи были у него сведены к числам.}
\alignedmarginpar{$^6$ Или: «и точно так же он разделял с ними мнение, что...».}
После указанных философских учений появилось исследование Платона, в большинстве вопросов примыкающее к пифагорейцам, а в некоторых отношениях имеющее свои особенности по сравнению с философией италийцев. Смолоду сблизившись прежде всего с Кратилом и гераклитовскими учениями, по которым все чувственные вещи находятся в постоянном течении и знания об этих вещах не существует, он здесь и позже держался таких воззрений. А так как Сократ занимался исследованием этических вопросов, а относительно всей природы в целом его совсем не вел, в названной же области искал всеобщего и первый направил свою мысль на общие определения, то Платон, усвоивши взгляд Сократа, по указанной причине признал, что такие определения имеют своим предметом нечто другое, а не чувственные вещи; ибо нельзя дать общего определения для какой-нибудь из чувственных вещей, поскольку вещи эти постоянно изменяются. Идя указанным путем, он подобные реальности $^1$ назвал идеями, а что касается чувственных вещей, то об них (по его словам) речь всегда идет $^2$ отдельно от идей и (в то же время) в соответствии с ними; ибо все множество вещей существует в силу приобщения к одноименным (сущностям). $^3$ При этом, говоря о приобщении, он переменил только имя: пифагорейцы утверждают, что вещи существуют по подражанию числам, а Платон — что по приобщению [изменивши имя]. $^4$ Но самое это приобщение или подражание идеям, что оно такое, — исследование этого вопроса было у них оставлено в стороне. Далее, помимо чувственных предметов и идей, он в промежутке устанавливает математические вещи, которые от чувственных предметов отличаются тем, что они вечные и неподвижные, а от идей — тем, что этих вещей имеется некоторое количество сходных друг с другом - сама же идея каждый раз только одна. И так как идеи являются причинами для всего остального, то их элементы он счел элементами всех вещей. Таким образом в качестве материи являются началами большое и малое, а в качестве сущности — единое; ибо идеи [они же числа] $^5$ получаются из большого и малого через приобщение (их) к единству. Что единое представляет собою сущность, а не носит наименование единого, будучи чем-либо другим, это Платон утверждал подобно пифагорейцам, и точно так же, как они $^6$ — что числа являются для всех остальных вещей причинами сущности (в них); а что он вместо неопределенного, как (чего-то) одного, ввел двоицу (пару) и составил неопределенное из большого и малого, это — его своеобразная черта;
\alignedmarginpar{$^7$ Некоторые ученые (Тренделенбург, Швеглер) под «первыми» числами, которые с платоновской точки зрения не могут быть выводимы из первичной «двойки», разумели числа идеальные, а Целлер считает, что слова «за исключением первых» должны быть удалены из текста, как позднейшая вставка. Но, как справедливо указывает Бониц (Comm. 94—95), речь идет здесь не об идеальных числах как таких (для конструкции которых и введена специально «двоица»), а о непроизводных математических числах: по Платону «двоица», есть принцип всякого повторения, и, следовательно, она бесполезна там, где число не образовано с помощью повторения (т.е. — какой-либо комбинации четных или нечетных чисел).}
кроме того, он полагает числа отдельно от чувственных вещей, а они говорят, что числа, это — сами вещи, и математические объекты в промежутке между теми и другими не помещают. Установление единого и чисел отдельно от вещей, а не так, как у пифагорейцев, и введение идей произошло вследствие исследования в области понятий (более ранние философы к диалектике не были причастны), а двоица (пара) была принята за другую основу потому, что числа, за исключением первых $^7$,легко выводились (рождались) из нее как из некоторой первичной массы. Однако же на самом деле происходит наоборот: то, что указывается здесь, не имеет хорошего основания. Эти философы из материи выводят множество, а идея (у них) рождает только один раз, между тем из одной материи, очевидно, получается один стол, а тот, который привносит идею $^8$, будучи один, производит много их (столов). Подобным же образом относится и мужское начало к женскому: это последнее заполняется через одно соитие, а мужское заполняет много (самок); и однако же это — подобия тех начал. $^9$

\alignedmarginpar{$^8$ Или, может быть, «то, что Платон называет «идеей».}
\alignedmarginpar{$^9$ Или: «и однако мы имеем здесь подобное же положение, как и при тех началах».}
\alignedmarginpar{$^{10}$ Т.е. с единым (причину добра) и с материальною двойственностью (причину зла). Дословно у Аристотеля: «причину добра и зла он указал в элементах — в каждом из двух одну из двух» (т.е. в одном — одну, в другом — другую).}
\alignedmarginpar{$^{11}$ Вместо «как, по нашим словам, искали ее...» можно было бы сказать: «мы напоминаем, что таким же образом ее искали» и т.д.}
Вот какие определения установил, таким образом, Платон по исследуемым (нами) вопросам. Из сказанного ясно, что он воспользовался только двумя причинами: причиною, определяющею суть вещи, и причиною из области материи (ибо идеи являются для всех остальных вещей причиною их сути, а для идей (таковою причиной является) единое); ясно также, что представляет собою материя, лежащая в основе (всего), — материя, которая получает определения через идеи при образовании чувственных вещей и (определения) через единое при образовании идей: эта материя есть двоица (пара), большое и малое. Кроме того, он связал с этими элементами $^{10}$ причину добра и причину зла, одну отнес к одному, другую — к другому, по образцу того, как, согласно нашим словам, искали ее $^{11}$ и некоторые из более ранних философов, например Эмпедокл и Анаксагор.

\newpage
\section{Книга 7}

\epigraph{
Разбор учений древних философов подтверждает, что, устанавливая четыре причины бытия, мы ни один род причин не пропустили. Материальное начало принимали все мыслители; некоторые добавляли также причину движения; философы, учившие идеям, з vrpoнули причину формальную; наконец, причина целевая никем не была сформулирована надлежащим образом, но в известном смысле о ней была речь у многих. Однако вне этих четырех родов причин какого-либо нового ни один философ не указал.
}{Парафраз А.В. Кубицкого}

\alignedmarginpar{$^1$ Здесь Аристотель, по-видимому, имеет в виду Анаксимандра, Беспредельное которого, образуя туманную неопределимую ближе массу, являлось первоисточником для всех элементов, принимавшихся в других теориях, и не сводилось ни к одному из таких элементов в отдельности.}
Мы проследили в сжатых чертах и (лишь) по главным вопросам, какие философы и как именно высказались относительно начал и истинного бытия; но во всяком случае мы имеем от них тот результат, что из говоривших о начале и причине никто не вышел за пределы тех (начал), которые были у нас различены в книгах о природе, но все явным образом так или иначе касаются, хоть и неясно, а все же (именно) этих начал. Они устанавливают начало в виде материи, все равно — взято ли у них при этом одно начало или несколько их и признают ли они это начало телом или бестелесным; так, например, Платон говорит о большом и малом, италийцы — о неопределенном, Эмпедокл — об огне, земле, воде и воздухе, Анаксагор — о неопределенном множестве подобночастных тел. Таким образом все эти мыслители воспользовались подобного рода причиной, и кроме того — все те, кто говорил о воздухе, или огне, или воде, или о начале, которое плотнее огня, но тоньше воздуха:
$^1$
ведь и такую природу приписали некоторые первоначальной стихии. Указанные под конец философы имели дело только с материальной причиной; а некоторые другие — (также) с той, откуда начало движения, как, например, все, кто делает началом дружбу и вражду, или ум, или любовь.

\alignedmarginpar{$^2$ «То, ради чего», — формальное обозначение для цели, блага (см. главу 3, пр. 1); в нашем случае — в дословном исходном значении.}
\alignedmarginpar{$^3$ Т.е. считают единое или сущее благом (имеются в виду Платон и его школа), как разбиравшиеся перед тем философы считали благом дружбу или ум. Бониц не прав, указывая (Comm. 98), что слова «приписывают такую (Бон. подчеркивает — «подобного рода») природу» имеют в виду непосредственно предшествующее — «что от них исходят движения».}
Затем, суть бытия и сущность отчетливо никто не указал, скорее же всего говорят (о них) те, кто вводит идеи; ибо не как материю принимают они идеи для чувственных вещей и единое — для идей, а также и не так, чтобы оттуда получалось начало движения (для вещей) (скорее у них это — причина неподвижности и пребывания в покое), но идеи для каждой из всех прочих вещей дают суть бытия, а для идей (это делает) единое. Что же касается того, ради чего
$^2$
происходят поступки, изменения и движения, то оно некоторым образом приводится у них в качестве причины, но не специально как цель и не так, как это следует. Ибо те, кто говорит про ум или дружбу, вводят эти причины как некоторое благо, но говорят о них не в том смысле, чтобы ради этих причин существовала или возникала какая-нибудь из вещей, а в том, что от них исходят (ведущие к благу) движения. Точно также и те, которые приписывают такую природу единому или сущему
$^3$
, считают его причиною сущности, но не принимают, чтобы ради него что-нибудь существовало или происходило. А поэтому у них получается, что они известным образом и вводят и не вводят благо как причину; ибо они говорят об этом не прямо, а на основании случайной связи.

\alignedmarginpar{$^4$ Я перевожу по тексту Ross'a, который, следуя Bywater'y, вместо τούτων пишет τοιούτον.}
Таким образом, что наши определения относительно причин даны правильно,— и сколько их, и какие они — об этом, по-видимому, свидетельствуют нам и все указанные мыслители, не имея возможности обратиться к (какой-либо) другой причине. Кроме того, ясно, что надо искать причины — или все так, как это указано здесь, или каким-нибудь подобным способом.
$^4$
А как высказался каждый из этих мыслителей и как обстоит дело относительно начал, возможные на этот счет затруднения мы теперь (последовательно) разберем.






\newpage
\section{Книга 8}

\epigraph{
Имея в виду проследить, в чем древние философы правильно судили о началах, в чем --- нет, Аристотель, в первую очередь разбирает учение тех, которые установили одну материальную причину. Затем подвергается критическому разбору учение Эмпедокла. Критикуется --- а в последовательно развитом виде до некоторой степени поддерживается --- учение Анаксагора. Все эти философы рассуждали только о чувственной природе вещей. Из числа философов, которые поставили себе задачей познавать вещи, не воспринимаемые чувствами, в первую очередь, подвергаются разбору учения пифагорейцев
}{Парафраз А.В. Кубицкого}

\alignedmarginpar{$^1$ Дословно: «имеющею причину»}
\alignedmarginpar{$^2$ Дословно: «рассматривая (изучая) природу в отношении всех вещей». Роз. и Первов переводят: «... рассуждая о природе всех вещей...»}
\alignedmarginpar{$^3$ Заключенные в скобки слова Christ выбрасывает из текста, считая их позднейшей вставкой, перебивающей ход мысли.}
\alignedmarginpar{$^4$ Точнее: «с наиболее мелкими частями и наиболее тонкое».}
\alignedmarginpar{$^5$ Дословно: «первою из тел».}
\alignedmarginpar{$^6$ Слово «популярный», по-видимому, не только дает дословный перевод, но и вполне точно передает тот оттенок, который здесь имеет у Аристотеля слово δημοτικός.}
 \alignedmarginpar{$^7$ Аристотель имеет в виду De coelo III, так что указание на «сочинения о природе» не всегда относится специально к «Физике».}
\alignedmarginpar{$^8$ При более точном (почти дословном) переводе — неправильная конструкция, как и в греческом тексте. «А про Анаксагора если бы кто счел, что он принимает два элемента, то он оказался бы в наибольшем соответствии с (
верным) ходом мысли, которого сам он, правда, не расчленил, но с необходимостью последовал бы за теми, кто стал бы указывать ему путь (склонять его за собой)».}
\alignedmarginpar{$^9$ Определенность «по существу» означает у Аристотеля определенность по первой из категорий (каковой у него является категория сущности), в отличие от определенности по качеству или количеству.}
Те, которые признают единство вселенной и вводят единую материальную основу, считая таковую телесной и протяженной $^1$, явным образом ошибаются во многих отношениях. В самом деле, они устанавливают элементы только для тел, а для бестелесных вещей — нет, в то время как есть и вещи бестелесные. Точно так же, начиная указывать причины для возникновения и уничтожения и рассматривая все вещи с натуралистической точки зрения $^2$, они упраздняют причину движения. Далее (они погрешают), не выставляя сущность причиною чего-либо, равно как и суть (вещи), и, кроме того, опрометчиво объявляют началом любое из простых тел, за исключением земли, не подвергнув рассмотрению, как совершается их возникновение друг из друга [я говорю здесь об огне, воде, земле и воздухе]. $^3$ В самом деле, одни вещи происходят друг из друга через соединение, другие — через разделение, а это различие имеет самое большое значение для вопроса, что — раньше и что — позже. При одной точке зрения наиболее элементарным могло бы показаться то, из чего как из первого вещи возникают через соединение, а таковым было бы тело с наиболее мелкими и тонкими частями. $^4$ Поэтому все, кто признает началом огонь, стоят, можно сказать, в наибольшем согласии с этой точкой зрения. И так же смотрит на элементарную основу тел и каждый из остальных (философов). По крайней море никто из числа тех, кто выступал позже и утверждал единство (первоосновы), не выставил требования, чтобы земля была элементом — очевидно, вследствие того, что (у нее) крупные части; а из трех (других) элементов каждый получил какого-нибудь заступника: одни ставят на это место огонь, другие — воду, третьи — воздух. И однако же почему они не указывают и землю, как это делает большинство людей? Ведь люди говорят, что все есть земля, да и Гесиод указывает, что земля произошла раньше всех тел $^5$: настолько древним и популярным $^6$ оказывается это мнение. Итак, если стоять на этой точке зрения, то было бы неправильно признавать (началом) какой-либо из этих элементов, кроме огня, и точно так же — считать, что такое начало плотнее воздуха, но разреженнее воды. Если же то, что позднее по происхождению, раньше по природе, а переработанное и смешанное — по происхождению позднее, то получается обратный порядок: вода будет раньше воздуха, а земля — раньше воды. Относительно тех, которые устанавливают одну такую причину, как мы указали, ограничимся сказанным. То же получается и в том случае, если кто устанавливает несколько таких причин, — как, например, Эмпедокл указывает, что материю образуют четыре тела: и у него также должны получиться частью то же самые, частью специальные затруднения.
\alignedmarginpar{$^{10}$ Аристотель здесь уже говорит языком своей философии, по которой принципом всякой определенности является та или иная форма, в противоположность неопределенности материи.}
\alignedmarginpar{$^{11}$ «Мы признаем» сказано от имени Платоновской школы, к которой Аристотель — в качестве одного из учеников Платона — часто себя причисляет (даже тогда, когда он специально занимается критикой платоновских учений, ср. А 9, 991 Ь 7).}
\alignedmarginpar{$^{12}$ Дословно: «К тому, что более представляется», по-видимому наиболее просто дополнить «правильным», «приемлемым». Стоящее кроме того в рукописях νυν (теперь) Bonitz и Christ опускают, так как оно не оговаривается у Alex., но слово это, по-видимому, вполне может быть сохранено, поскольку оно специально указывает на современные (может быть, даже на аристотелевские) взгляды.}
\alignedmarginpar{$^{13}$ Т.е. лишь относительно сущности, подверженной возникновению, уничтожению и движению.}
\alignedmarginpar{$^{14}$ Для передачи общего хода развития мысли фразу можно сформулировать так: «Прежде всего так называемые пифагорейцы пользуются более необычными началами и элементами...» (этому «прежде всего», по-видимому, соответствует затем начало главы 9).}
\alignedmarginpar{$^{15}$ Дословно: «физиологи» (см. примечание 13 к 5-й гл. I книги).}
\alignedmarginpar{$^{16}$ Я перевожу «астрономии», чтобы избежать связавшегося у нас с термином «астрология» отрицательного оттенка; у греков употреблялись оба термина — Аристотель говорит почти исключительно «астрология», Платон в «Государстве» «астрономия» (Politeia VII, главы 10—11).}
\alignedmarginpar{$^{17}$ Дословно: «они нисколько не больше относят свои слова к математическим телам, чем к чувственно-воспринимаемым». Иными словами: если из предела и беспредельного еще можно с натяжкою получить протяженность, то как же из них вывести тяжесть? А между тем эти начала должны у них все объяснять одинаково, — математические свойства тел не больше, чем физические. — Таким образом, Швеглер и Бониц правы, признавая ненужной перестановку Казавбона, который хотел читать: «они не говорят о чувственных вещах ничего больше, чем о математических».}
В самом деле, мы видим, что элементы возникают друг из друга, так что огонь и земля не всегда остаются тем же самым телом (об этом сказано в сочинениях о природе $^7$); а относительно причины движущихся тел, принимать ли одну такую причину или две, об этом, надо считать, (у него) совсем не сказано сколько-нибудь правильно или обоснованно. Вообще у тех, кто говорит таким образом, необходимо упраздняется качественное изменение; ибо не может (у них) получиться ни холодное из теплого, ни теплое из холодного. В самом деле, (тогда) чему-нибудь должны были бы принадлежать сами эти противоположные свойства и должно было бы существовать какое-нибудь одно вещество, которое становится огнем и водой, — а он отрицает это.

Что касается (теперь) Анаксагора, то если сказать, что он принимает два элемента, это наиболее соответствовало бы правильному ходу мысли, причем сам он его, правда, не продумал, но необходимо последовал бы за теми $^8$, кто стал бы указывать ему (надлежащий) путь. В самом деле, (я уже не говорю о том, что) (по ряду причин) нелепо утверждать изначальное смешение всех вещей, — и потому, что они в таком случае должны были бы ранее существовать в несмешанном виде; и потому, что от природы не свойственно смешиваться чему попало с чем попало; а кроме того и потому, что отдельные состояния и привходящие (случайные) свойства отделялись бы (в таком случае) от субстанций (одно и то же ведь подвергается смешению и отделению): но при всем том, если бы последовать (за ним), анализируя вместе (с ним) то, что он хочет сказать, то его слова произвели бы более современное впечатление. В то время, когда ничего не было выделено, об этой субстанции, очевидно, ничего нельзя было правильно сказать; я имею в виду, например, что она не была ни белого, ни черного, ни серого или иного цвета, но необходимо была бесцветной, — иначе у нее был бы какой-нибудь из этих цветов. И подобным же образом она была без вкуса на этом же самом основании, и у нее не было никакого другого из подобных (свойств). Ибо для нее невозможно иметь ни качественную определенность, ни количественную, ни определенность по существу $^9$: в таком случае у нее была бы какая-нибудь из так называемых частичных форм $^{10}$, а это невозможно, раз все находилось в смешении; тогда уже произошло бы выделение, а между тем он утверждает, что все было смешано, кроме ума, и лишь он один — несмешан и чист. Из сказанного для Анаксагора получается, что он в качестве начал указывает единое (оно ведь является простым и несмешанным) и «иное» (это последнее — в том смысле), как мы признаем $^{11}$ неопределенное — до того как оно получило определенность и приняло какую-нибудь форму. Таким образом он говорит неправильно и неясно, но в своих намерениях приближается к мыслителям, выступающим позднее, и к более принятым теперь взглядам. $^{12}$

Эти философы, однако, близко занялись только рассуждениями о возникновении, уничтожении и движении: и начала и причины они исследуют почти исключительно в отношении такого рода сущности. $^{13}$ Те же, которые подвергают рассмотрению всю совокупность бытия, а в области бытия различают, с одной стороны, чувственные, с другой — нечувственные вещи, (такие мыслители), очевидно, производят исследование обоих (этих) родов, и поэтому можно бы более обстоятельно заняться ими,
\alignedmarginpar{$^{18}$ Начинающееся отсюда крайне трудное место (990 а 18—29), истолковать которое в свое время отказался Бониц, теперь можно считать в значительной мере разъясненным совместными усилиями ученых. Аристотель хочет показать то своеобразное затруднение, которое получается для пифагорейцев, поскольку они отождествляют вещи с числами, а не считают числа образцами для вещей, как это делает Платон. Вселенная есть число или некоторая совокупность чисел; отдельные части вселенной — отдельные числа, которые реализованы в этих частях. Поэтому, считая сущностью различных вещей те или другие числа, пифагорейцы помещали эти вещи в те части вселенной, которые являли собою реализацию данных чисел (например, «мнение» помещалось на землю, потому что число мнения есть два, и в то же время два есть число земли, вернее — число группы, состоящей из земли и центрального огня). Таким образом, естественно встает вопрос, имеет ли (если даже держаться пифагорейской концепции) значение для той или другой вещи (мнения, несправедливости и т.д.) осуществление составляющего эту вещь числа в каком-нибудь определенном месте вселенной (как это странным образом выходило у пифагорейцев, совмещавших предметы совершенно разнородные), или же истинною природой всякой вещи они должны бы были считать логическое существо ее числа, независимо от космической роли этого числа (ведь и пифагорейцы учили, что вещи подражают числам, и, следовательно, разные вещи могли выявлять собою одни и те же числа, но с разных точек зрения, — или непосредственно имея эти числа своею сущностью, или представляя собою конкретное космическое их отображение).}
\alignedmarginpar{$^{19}$ Ross (I 185—186), давая здесь несколько иное объяснение, предлагает толковать «составных <материальных> величин», имея при этом в виду элементарные тела (стихии), «составляемые» из того или другого числа и находящиеся в разных местах мира.}
\alignedmarginpar{$^{20}$ Так — по конъектуре Целлера, предложившего здесь читать διό вместо oti τό. Однако же, и в случае оставления засвидетельствованного рукописями текста (Christ и Ross) получается, может быть, достаточно удовлетворительный смысл: «... вследствие того, что указанные явления и т.д.». В таком случае уже — обратный порядок: в определенном месте появляется определенное количество небесных тел, потому что оно отмечено определенным числом и наличием соответствующих вещей (или явлений).}
\alignedmarginpar{$^{21}$ Проще и достаточно точно по смыслу будет перевести: «то надо ли признать, что здесь мы имеем это же самое находящееся на небе число, которое доставляет каждое данное явление и т.д.».}
(определяя), что сказано у них удачно или неудачно для выяснения стоящих теперь перед нами вопросов.
Что касается так называемых пифагорейцев, то они пользуются $^{14}$ более необычными началами и элементами, нежели философы природы $^{15}$ (причина здесь — в том, что они к началам этим пришли не от чувственных вещей; ибо математические предметы чужды движению, за исключением тех, которые относятся к астрономии $^{16}$); но при этом все свои рассуждения и занятия они сосредоточивают на природе. В самом деле, они построяют небо и прослеживают то, что получается для его частей, состояний и действий, и на это используют свои начала и причины,
как бы соглашаясь с другими натурфилософами, что бытием является (лишь) то, что воспринимается чувствами и что объемлет так называемое небо. Однако же, как мы сказали, причины и начала, которые они указывают, достаточны для того, чтобы подняться и в более высокую область бытия, и более подходят (для этого), нежели для рассуждений о природе. Но, с другой стороны, откуда получится движение, когда в основе лежат только предел и беспредельное, нечетное и четное, — об этом они ничего не говорят, и вместе с тем (не указывают) — как возможно, чтобы без движения и изменения происходили возникновение и уничтожение, или действия несущихся по небу (тел).

Далее, если бы и признать вместе с ними, что из этих начал (т.е. предела и беспредельного) образуется величина, или если бы было доказано это, — все же каким образом получится, что одни тела — легкие, а другие — имеют тяжесть? В самом деле, исходя от тех начал, которые они кладут в основу и указывают, они не в меньшей мере берутся давать разъяснения относительно чувственных тел, чем относительно математических $^{17}$; в соответствии с тем об огне, земле и других таких телах у них совсем ничего не сказано, думаю — потому, что в отношении чувственных вещей они никаких специальных указаний не давали. Далее $^{18}$, как это понять, что свойства числа и (само) число являются причиной того, что и изначала и в настоящее время существует на небе и совершается в нем, а (вместе с тем) нет никакого другого числа, кроме того, из которого состоит вселенная? Если у них в такой-то части (мира) находится мнение и (в такой-то) удача, а немного выше или ниже — несправедливость и отделение или смешение, причем в доказательство тому они приводят, что каждое из этих явлений есть число, а в данном месте оказывается уже (именно) это количество находящихся рядом (сосуществующих) (небесных) тел $^{19}$, вследствие чего $^{20}$ указанные явления сопутствуют каждый раз соответственным местам; (если, таким образом, у них устанавливается тесная связь между отдельными явлениями и числами, господствующими в разных частях мира), —то будем ли мы иметь здесь (по отношению к явлениям) тоже самое находящееся на небе число, про которое надлежит принять, что оно составляет каждое данное явление, или же здесь будет не это (мирообразующее) число, а другое $^{21}$? Платон (во всяком случае) говорит, что — другое: правда, и он считает числами и вещи и причины вещей, но причинами он считает числа умопостигаемые, а те, которые отождествляются с вещами, воспринимаются чувствами.

\vfill




\newpage
\section{Книга 9}

\epigraph{
Имея в виду подвергнуть крптике учение Платона, Аристотель отмечает, что гипотеза, устанавливающая идеи, опрометчиво удваивает число вещей, которые надо объяснить; затем он в первую очередь разбирает аргументы, которые Платон выдвигал в пользу существования идеи·, и констатирует в них различные логические дефекты: в одних вывод не следует с (силлогистическою) необходимостью, другие идут дальше, чем хотел сам философ. Далее он разбирает взаимоотношение между идеями и чувственнымн вещами и вопрос о том, какое воздействие производят идеи на чувственные вещи. После этого он переходит к природе чисел (к которым Платон пытался свести идеи) и излагает те трудности, которые в этом случае получаются как при попытках формулировать существо идей --- чисел, так и при объяснении из них вещей. В результате Аристотель делает вывод, что в платоновской школе весь вопрос был поставлен самым превратным образом: были отвергнуты вещи и причины, непосредственно данные, и философы установили сокровенные и совершенно бесплодные начала; их доказательства не доказывают, чего они хотят, и природу геометрических объектов нельзя совместить с характером всего учения в целом. Наконец, в виду того, что они установили одни и те же начала для всех получается, что их нельзя ни познавать, ни считать прирожденными человеку от природы
}{Парафраз А.В. Кубицкого}

\alignedmarginpar{Примечания к этой главе находятся в конце этой главы.}
Относительно пифагорейцев (вопрос) теперь оставим: их достаточно коснуться в такой мере. Что же касается тех, которые в качестве причин устанавливают идеи, то прежде всего они, стремясь получить причины для здешних вещей, ввели другие предметы, равные этим вещам по числу, как если бы кто, желая произвести подсчет, при меньшем количестве вещей думал, что это будет ему не по силам, а, увеличив (их) количество, стал считать. В самом деле, идей приблизительно столько же или (уж) но меньше, чем вещей, для которых искали причин, причем эти поиски $^1$ и привели от вещей к идеям; ибо для каждого (рода) есть нечто одноименное $^2$, (есть оно) — помимо сущностей — и для всего другого, где имеется единое, относящееся ко многому, и в области здешних вещей и в области вещей вечных. Далее, если взять те способы, которыми мы $^3$ доказываем существование идей, то ни один из них не устанавливает с очевидностью (такого существования): на основе одних не получается с необходимостью силлогизма $^4$, на основе других идеи получаются и для тех объектов, для которых мы (их) не утверждаем. В самом деле, по «доказательствам от наук» $^5$, идеи будут существовать для всего, что составляет предмет науки, на основании «единого относящегося ко многому» $^6$, (получатся идеи) и для отрицаний, а на основании «наличия объекта у мысли по уничтожении (вещи)» $^7$— для (отдельных) преходящих вещей (как таких): ведь о них имеется (у нас) некоторое представление. Далее, из более точных доказательств $^8$ одни устанавливают идеи отношений, для которых, по нашим словам, не существует отдельного самостоятельного рода $^9$, другие утверждают «третьего человека».$^{10}$ И, вообще говоря, доказательства, относящиеся к идеям, упраздняют то, существование чего нам [сторонникам идей] важнее, нежели существование идей: выходит, что не двоица является первой, но число, и что существующее в отношении раньше («первее»), чем существующее в себе, и сюда же принадлежат все те вопросы, по которым отдельные (мыслители), примкнувшие к взглядам относительно идей, пришли в столкновение с основными началами (этого учения).

Далее, согласно исходным положениям, на основании которых мы утверждаем идеи, должны существовать не только идеи сущностей, но и идеи многого другого (в самом деле, и мысль есть одна (мысль), не только когда она направлена на сущности, но и по отношению ко всему остальному, и науки имеют своим предметом не только сущность, но и другого рода бытие, и можно выдвинуть несметное число других подобных (соображений)); между тем по (логической) необходимости и (фактически) существующим относительно идей взглядам, раз возможно приобщение к идеям, то должны существовать только идеи сущностей; ибо приобщение к ним не может носить характера случайности, но по отношению к каждой идее причастность должна иметь место постольку, поскольку эта идея не высказывается о подлежащем $^{11}$ (так например, если что-нибудь причастно к двойному в себе, оно причастно и к вечному, но — через случайное соотношение; ибо для двойного быть вечным — случайно). Таким образом, идеи будут (всегда) представлять собою сущность.$^{12}$
А у сущности одно и то же значение и в здешнем мире и в тамошнем. Иначе какой (еще) может иметь смысл говорить, что есть что-то помимо здешних вещей, — единое, относящееся ко многому? И если к одному и тому же виду (группе) $^{13}$ принадлежат идеи и причастные им вещи, тогда (между ними) будет нечто общее (в самом деле, почему для преходящих двоек и двоек многих, но вечных $^{14}$ существо их как двоек в большой мере одно и то же, чем для двойки самой по себе, с одной стороны, и какой-нибудь отдельной двойки — с другой?) Если же здесь не один и тот же вид $^{15}$ бытия, то у них было бы только одно имя общее, и было бы похоже на то, как если бы кто называл человеком и Каллия и кусок дерева, не усмотрев никакой общности между ними.

Однако в наибольшее затруднение поставил бы вопрос, какую же пользу приносят идеи по отношению к воспринимаемым чувствами вещам, — тем, которые обладают вечностью, или тем, которые возникают и погибают. Дело в том, что они не являются для этих вещей причиною какого-либо движения или изменения. А с другой стороны, они ничего не дают и для познания всех остальных предметов (они ведь и не составляют сущности таких предметов, — иначе они были бы в них), и точно так же (они бесполезны) для их бытия, раз они не находятся в причастных к ним вещах. Правда, можно было бы, пожалуй, подумать, что они являются причинами таким же образом, как белое, если его подмешать, (является причиной) для белого предмета. Но это соображение — высказывал его прежде всех Анаксагор, а потом Евдокс и некоторые другие — представляется слишком уж шатким: нетрудно собрать много невозможных последствий против такого взгляда. А вместе с тем и из идей (как таких) $^{16}$ не получается остального бытия ни одним из тех способов, о которых (здесь) обычно идет речь. $^{17}$ Говорить же, что идеи это — образцы и что все остальное им причастно, это значит произносить пустые слова и выражаться поэтическими метафорами. В самом деле, что это за существо, которое действует, взирая на идеи? Можно и быть и становиться сходным с чем угодно $^{18}$, в то же время и не представляя копии с него; так что и если есть Сократ и если нет его, может появиться такой же (человек), как Сократ; и подобным же образом, очевидно, (было бы) и в том случае, если бы Сократ был вечным. Точно так же будет несколько образцов у одной и той же вещи, а значит — и (несколько) идей, например для человека — живое существо и двуногое, а вместе с тем — и человек в себе. Далее, не только для воспринимаемых чувствами вещей являются идеи образцами, но также и для них самих, например род, как род, для видов; так что одно и то же будет и образцом и копией (другого образца). Далее, покажется, пожалуй $^{19}$, невозможным, чтобы врозь находились сущность и то, чего она есть сущность; поэтому как могут идеи, будучи сущностями вещей, существовать отдельно (от них)? Между тем в «Федоне» высказывается та мысль, что идеи являются причинами и для бытия, и для возникновения (вещей); и однако же при наличии идей вещи, (им) причастные, все же не возникают, если нет того, что произведет движение; и возникает многое другое, например дом и кольцо, для которых мы идей не принимаем; а потому ясно, что и все остальное $^{20}$ может и существовать и возникать вследствие таких же причин, как и вещи, указанные сейчас.$^{21}$


Далее, если идеи представляют собою числа, то как будут они выступать в качестве причин? Потому ли, что (сами) вещи $^{22}$, это — другие числа, например это вот число — человек, это — Сократ, а это — Каллий? Тогда в каком смысле образуют те (идеальные) числа причины для этих? Ведь если и (считать, что) одни — вечные, а другие — нет, это никакой разницы не составит. Если же (идеи — числа являются причинами), потому что здешние вещи представляют отношения чисел — таково, например, созвучие, — тогда, очевидно, существует некоторая единая основа (для всех тех составных частей) $^{23}$, отношениями которых являются эти вещи. Если есть какая-нибудь такая основа, (скажем) материя $^{24}$, то очевидно, что и числа сами в себе $^{25}$ будут известными отношениями одного к другому. Я хочу сказать, например, что если Каллий есть (выраженное) в числах отношение огня, земли, воды и воздуха, тогда и идея будет числом каких-нибудь других лежащих в основе вещей $^{26}$; и человек в себе — все равно выражен ли он каким-нибудь числом или нет — все же будет (по существу дела) отношением в числах каких-нибудь вещей, а не числом $^{27}$, и не будет на этом основании существовать какого-либо числа (в себе). $^{28}$


Далее, из нескольких чисел получается одно число, а из идей $^{29}$ как может получиться одна идея? $^{30}$ Если же (новые образования получаются) не из самих идеальных чисел, а из единиц, находящихся в составе числа, например в составе десяти тысяч, то как обстоит дело с (этими) единицами? Если они однородны, получится много нелепостей $^{31}$, точно так же — если они неоднородны, ни — сами единицы (находящиеся в числе) — друг с другом, ни — все остальные между собой. $^{32}$ В самом деле, (в этом последнем случае) чем будут они отличаться (друг от друга), раз у них (вообще) нет свойств? И не обосновано это, и не согласуется с требованиями мысли. Кроме того, оказывается необходимым устанавливать еще другой род числа, с которым имеет дело арифметика, и также все то, что у не которых получает обозначение промежуточных (объектов); так вот, эти объекты — как они существуют, или из каких образуются начал? а также — почему они будут находиться в промежутке между здешними вещами и (числами) самими по себе? Затем, если взять единицы, которые находятся в двойке, то каждая из них образуется из некоторой предшествующей двойки; $^{33}$ однако же это невозможно. Далее, почему образует единство получаемое через соединение число? $^{34}$ Далее, помимо того, что (уже) было сказано (раньше), если единицы различны (между собой), то надо было говорить по образцу тех, которые признают, что элементов — четыре или два: ведь и каждый из них не дает имя элемента тому, что (здесь) есть общего, например, телу, а огню и земле, независимо от того, имеется ли (при этом) нечто общее, а именно тело, или нет. Теперь же дело ставится таким образом, будто единое, подобно огню или воде, состоит из однородных частей; а если так, то числа не будут сущностями; напротив, ясно, что если имеется некоторое единое в себе и оно является началом, то, значит, единое имеет несколько значений $^{35}$; иначе быть не может. Кроме того $^{36}$, желая сущности возвести к началам, мы $^{37}$ длины $^{38}$ выводим из короткого и длинного, из некоторого малого и большого, плоскость — из широкого и узкого, а тело — из глубокого и низкого. $^{39}$ И однако, как (в таком случае) $^{40}$ будет плоскость вмещать линию, или объем — линию и плоскость: ведь к разным родам относятся широкое и узкое (с одной стороны), глубокое и низкое — (с другой). Поэтому как число не будет находиться в них $^{41}$, потому что многое и немногое отличны от этих начал $^{42}$, так точно, очевидно, и никакое другое из высших определений не будет входить в состав низших. Кроме того, и родом не является широкое по отношению к глубокому, иначе тело было бы некоторою плоскостью. Далее, откуда получатся точки в том, в чем они находятся? Правда, с этим родом (бытия) и боролся Платон как с (чисто) геометрическим учением, а применял название начала линии, и часто указывал он на это — на «неделимые линии». $^{43}$ Однако же необходимо, чтобы у [этих] линий был какой-то предел. Поэтому на том же основании, почему существует линия, существует и точка.

Вообще, в то время как мудрость ищет причину открывающихся нашему взору вещей, мы $^{44}$ этот вопрос оставили в стороне (мы ведь ничего не говорим о причине, откуда берет начало изменение), но, считая, что мы указываем сущность этих вещей, (на самом деле) утверждаем существование других сущностей; а каким образом эти последние являются сущностями наших (здешних) вещей, об этом мы говорим по пустому; ибо причастность (как мы и раньше сказали) не означает ничего. Равным образом, что касается той причины, которая, как мы видим, имеет (основное) значение для наук $^{45}$, — той, ради которой творит всякий разум и всякая природа, — к этой причине, которую мы признаем одним из начал, идеи также никакого отношения не имеют, но математика стала для теперешних (мыслителей) философией, хотя они говорят что ею нужно заниматься ради других целей. Далее, относительно сущности, которая (у платоновцев) лежит в основе как материя, можно бы признать, что она имеет слишком математический $^{46}$ характер и, сказываясь о сущности и материи, скорее образует отличительное свойство той и другой, нежели материю $^{47}$ ; именно так обстоит с большим и малым, подобно тому как и исследователи природы («физиологи») говорят о редком и плотном, признавая их первыми отличиями основного вещества; ибо это есть некоторый избыток и недостаток. $^{48}$ И что касается движения, — если в указанных сейчас свойствах будет (корениться) движение $^{49}$, тогда, очевидно, идеи будут двигаться; если же нет, откуда движение появилось? (В таком случае) все исследование природы оказывается упраздненным. Также и то, что представляется легким делом, — доказать, что все едино, — (на самом деле) не удается; ибо через вынесение (общего) не становится все единым, но получается некоторое единое в себе, если (даже) принять все (предпосылки). $^{50}$ Да и этого (единого в себе) не получается, если не признать, что всеобщее является родом $^{51}$; а это в некоторых случаях невозможно. — Не дается никакого объяснения и для того, что (у них) идет за числами, — для длин, плоскостей и тел, ни — тому, как они существуют или должны существовать, ни — тому, в чем их значение; это не могут быть ни идеи (они ведь не — числа), ни промежуточные вещи (таковыми являются математические объекты), ни — вещи преходящие, но здесь, по-видимому, опять какой-то другой — четвертый род (сущего).

Вообще, если искать элементы того, что существует, не произведя предварительных различений, то ввиду большого количества значений у сущего $^{52}$ найти (ответ) нельзя, особенно, когда вопрос ставится таким образом: из каких элементов оно (сущее) состоит? В самом деле, из чего состоит действие или страдание, или прямое, получить (указание), конечно, нельзя, а если возможно, то лишь в отношении сущностей.  А потому искать элементы всего, что существует, или думать, что имеешь их, не соответствует истине. — Да и как было бы возможно познать человеку элементы всех вещей? Ведь ясно, что до этого (познания) он раньше знать ничего не может. $^{53}$ Как тому, кто учится геометрии, другие вещи раньше знать возможно, а чем занимается эта наука и о чем он имеет получать познания, этого он заранее совсем не знает; так именно обстоит дело и во всех остальных случаях. Поэтому, если есть какая-нибудь наука обо всем существующем, как утверждают некоторые, то такой человек не может раньше (ее) знать что бы то ни было. А между тем всякое изучение происходит через предварительное знание или всех (исходных данных), или некоторых, — и то, которое орудием имеет доказательство, и то, которое обращается к определениям; ибо составные части определения надо знать заранее, и они должны быть (нам) понятными $^{54}$; и то же имеет силу и для изучения через индукцию. С другой стороны, если бы даже оказалось, что нам такое знание прирождено $^{55}$, то нельзя не изумляться, как это мы, сами того не замечая, обладаем наилучшею из наук. — Далее, как можно будет узнать, из каких именно элементов состоит сущее $^{56}$, и как это станет ясным? В этом тоже ведь есть затруднение. В самом деле, здесь можно спорить так же, как и о некоторых слогах: одни говорят, что "ζα" состоит из "с", "д" и "я", а некоторые утверждают, что это другой звук, отличный от всех известных (нам). $^{57}$

Кроме того, в отношении вещей, которые подлежат чувственному восприятию, как может их кто-нибудь знать, не имея этого восприятия? И однако же это было бы необходимо, раз все вещи состоят из одних и тех же элементов $^{58}$, подобно тому как сложные звуки состоят из элементов, свойственных (этой) области.



\subsection{Примечания к главе 9}

$^1$ Дословно: «для которых ища причин, от них пришли к идеям».

$^2$ Я сохраняю текст, установленный Christ'ом, ибо текст Ross'a (καέχαστον γάρ δμώνυ[Λ0ν τι έστι και παρά τάς ουσίας, των τε άλλων εστίν εν έπι πολλών) — «в отношении к каждой вещи (или, может быть, к каждому роду вещей) есть нечто одноименное и помимо сущностей, и для всего остального есть единое во многом» — повторяет второй фразой первую, если не толковать «и помимо сущностей» весьма натянутым образом («есть нечто одноименное и существующее помимо <здешних> сущностей»), на что, однако, у Ross'a как будто есть намек («an entity existing apart from the substances»), Ross I 191).

$^3$ Как поясняет Alex. (Comm. 58, 15 —16), «слова «мы доказываем» обнаруживают, что, передавая суждения Платона, он говорит, как против близкого (родного) мнения».

$^4$ Alex. (ib. 58, 27—29) приводит такие примеры «аргументов, из которых не получается силлогизма»: «если есть какая-нибудь истина, то надо думать, что существуют идеи, ибо из здешних вещей ничто не истинно; и если есть память, то есть идеи, ибо намять имеет своим предметом то, что пребывает».

$^5$ Отдельные аргументы в пользу идей, очевидно, постепенно получили в школе Платона сокращенные условные обозначения, и Аристотель, обращаясь в своем «докладе» к товарищам по школе — именно такой характер, по-видимому, первоначально носила составившая затем первую книгу всей «Метафизики» работа, — указывает те или другие из этих аргументов посредством их общеизвестных «школьных» наименований. В частности, «доказательства от наук» исходили от утверждения, что объект науки должен быть устойчивым и носить общий характер, между тем у чувственных предметов этих свойств нет; таким образом, выдвигается требование особого, отдельного от чувственных вещей предмета. Аристотель подчеркивает, что при этих условиях все рассматриваемое науками общее должно было превратиться в идеальные предметы; между тем в ряде случаев общее остается общим, не получая гипостазирования (так — в отношении всех объектов техники фактическую реальность — и с точки зрения Платоновской школы — можно приписывать только индивидуальному).

$^6$ «Единое, относящееся ко многому», приводит к неприемлемым для Платоновской школы выводам в том, например, случае, если одно и то же определение отрицается в отношении ряда предметов. Так, например, «не человек» или «не образованный» есть общее обозначение, высказываемое относительно множества объектов (лошади, собаки и т.д.), но такой идеи школа не устанавливает.

$^7$ Указывая на «наличие объекта у мысли по уничтожении вещи», Платон требовал признания идей, поскольку, независимо от уничтожения ряда преходящих объектов, мысль человека продолжает иметь перед собою некоторое содержание, общее всему — пусть даже всецело уничтожившемуся — ряду. Но если объективировать такое содержание, то на этом же основании приходится требовать также объективации содержания всякого индивидуального предмета, поскольку это содержание сохраняется в той или иной мысли — уже по уничтожении данного предмета, и, следовательно, будет столько идей, сколько было индивидуальных вещей.

$^8$ Ross (I, 194), со ссылкой на Джэксона, отмечает, что в предшествующем анализе Аристотелем были указаны противоречащие платоновскому учению выводы, сделанные Аристотелем из платоновских аргументов в пользу идей; теперь Аристотель констатирует затруднения, связанные с более разработанными доказательствами, — затруднения, которые сам Платон сознавал и которые, не создавая внутренних противоречий, все же побуждали школу к нежелательным для нее выводам.

$^9$ Ross приписывает здесь Аристотелю мысль, что с точки зрения Платоновской школы нельзя выделить соотнесенных предметов в одну самостоятельную группу («мы — имеются в виду платоновцы — не предполагаем, что все вещи, которые случайно оказываются равными другим вещам, образуют отдельный класс среди вещей (in rerum natura)», I 194). На самом деле, по-видимому, Аристотель говорит не о том, что школа не считала возможным создавать из соотнесенных предметов самостоятельную группу, а о том, что по ее основному взгляду отношения не могут образовать такую группу, — группу, которая должна бы была мыслиться отдельно от соотнесенных предметов.

$^{10}$ Аргумент относительно «третьего человека», выдвигавшийся еще непосредственно против Платона и учтенный им в диалоге «Парменид», требовал принимать идею, общую ряду однородных индивидуальных вещей, с одной стороны, и идее ряда этих вещей — с другой. Таким образом, оказывалось необходимым продолжить в бесконечность тот путь, который первоначально объединил в одной идее ряд индивидуальных вещей.

$^{11}$ Если некоторая идея (например, вечности) не существует самостоятельно, а есть только свойство другого — самостоятельного — бытия («высказывается о подлежащем»), тогда в этой области все вещи надо было бы в первую очередь ставить в зависимость от того основного бытия, коего наша идея является свойством, и они определялись бы через природу этого бытия, а к нашей идее находились бы только в случайном отношении, т.е. выражаемое ею свойство отнюдь не было бы им с необходимостью присуще, — так, например, если мы в качестве самостоятельного бытия возьмем двойное в себе, тогда вещи, причастные двойному, отнюдь еще не должны быть (а только иногда случайно могут быть) вечными, хотя вечность и есть одно из свойств двойного в себе.

$^{12}$ Бониц предлагает здесь внести изменение в текст рукописей и переводит это место: «Таким образом, идеи будут идеями сущностей» (ώστ'έσται ουσιών τά εΐδη), считая, что Аристотель уже в этот момент приходит к намеченному им выводу («по логической необходимости... должны существовать только идеи сущностей»). Christ и Ross не принимают в свой текст Вonitz'евской конъектуры, и она не будет нужна, если признать, что в данный момент цель Аристотеля — только подчеркнуть необходимость для идей быть сущностями, поскольку они — предмет приобщения (Bon. исходит из предположения, что идеи — всегда сущности). Общий ход мысли Аристотеля, по-видимому, такой: «Идеи должны быть только идеями сущностей. Это вытекает из того, что вещи к ним причастны. Дело в том, что вещи должны быть причастны к идеям самим по себе (в их самостоятельном бытии), а не поскольку они — свойства другого (не поскольку они «высказываются о подлежащем»), ибо в этом последнем случае определение вещи через идею было бы случайным, вещь основным образом определялась бы через то другое, о чем «сказывается» такая идея, и выражаемое идеею свойство могло бы и не принадлежать вещи, вещь находилась бы к этой идее в случайном отношении. Значит, идеи, поскольку им причастны вещи, должны быть сущностями. А отсюда (это уже следующий шаг) — вещи, причастные идеям (как сущностям), также должны быть сущностями. — И только при этом условии (если те и другие, как сущности, принадлежат к одному роду) между ними может быть действительная связь, в противном случае между ними будет общность только по имени. — Таким образом, та связь, которая первоначальными аргументами устанавливалась между всякого рода множеством и объединяющей его идеей, теперь объявляется совершенно нереальною.»

$^{13}$ είδος — здесь опять в смысле — группа, разряд (как и в выражении έν ύλης ehsi). У идей и причастных им вещей будет один είδος, поскольку «один и тот же смысл имеет сущность и здесь и там» (см. выше).

$^{14}$ Имеется в виду сопоставление всякого рода чувственно-воспринимаемых пар, с одной стороны, и математических двоек, которых может быть неопределенно много совершенно одинаковых, с другой (см. выше 987 Ь 16—17).

$^{15}$ В том смысле, как указано выше, в примечании 13.

$^{16}$ Т.е. поскольку они выступают не в качестве имманентных начал (как это было у Анаксагора и Евдокса), а в собственном — платоновском — смысле, в качестве начал трансцендентных.

$^{17}$ Аристотель имеет в виду главу 24 книги V, где указаны различные значения для выражения «быть из чего-нибудь», и констатирует, что ни одно из этих значений не годится, чтобы формулировать отношение между идеями и здешним бытием.

$^{18}$ Я читаю по указываемой у Ross'а конъектуре Richards'а α 24 ότωουν («с чем угодно»), вместо рукописного и принятого в изданиях ότιοον («... что угодно может и быть и становиться сходным»), так как иначе дальнейшее πρός έεΐνο («с него») остается без объекта, к которому оно бы относилось.

$^{19}$ Или: не сочтем ли мы (невозможным).

$^{20}$ Т.е. естественно существующие вещи (τά φύσει).

$^{21}$ Вещи, создаваемые искусством (τά τέχντ,).

$^{22}$ Коих идеи — числа должны быть причинами.

$^{23}$ Чисто грамматически здесь конструкция представляется неправильной (δήλον δτι εστίν ένγέ τ», ών είσι λόγοι <τάνταύθα>), и поэтому Walker, как отмечает Ross, предложил читать ου вместо ών. Однако, Ross считает здесь множественное число приемлемым и даже более желательным, указывая, что отношения требуют двух соотносящихся членов, и все место объясняет так: очевидно, что к чему-то одному сводятся (Ross дословно говорит — «какой-нибудь один класс вещей образуют») те элементы (у Ross'a— the things I 200), между которыми здешние вещи представляют собою то или другое отношение.

$^{24}$ Я читаю по тексту Christ'а εί oh τι-τούτο, ή υλη, имеющему хорошую рукописную базу и по смыслу здесь заслуживающему предпочтения перед текстом Ross'a ει δή τούτο ή ΰλη (6 14—15).

$^{25}$ Т.е. идеальные числа.

$^{26}$ Как указывают и Bon. и Ross, здесь по ходу аргументации естественно было бы сказать: «идея будет отношением в числах каких-нибудь других... вещей». Но Ross вместе с тем отмечает, что рукописный текст можно удержать, потому что в данном месте центр тяжести — в словах «каких-нибудь других лежащих в основе вещей», и Аристотель здесь называет идею числом, еще пользуясь пока терминологией Платона (или, может быть, поскольку число будет уже в этом случае выражением отношения между числами).

$^{27}$ Идеи, если они — числа, всегда должны, как и здешние вещи, представлять числовое отношение тех или других субстратов. При этом в одних случаях они могут выступать как числа (например, число октавы — 2, потому что отношение октавы 2:1), в других — как отношения чисел (например, число мяса или кости — 3 части огня, 2 части земли, см. Met. XIV, 5), но всегда в основе своей будут именно отношениями.

$^{28}$ Эти последние слова, по-видимому, (ср. Ross I 200) специально означают: и из-за того, что некоторые идеи носят характер чисел, не следует, чтобы в идеальном мире существовали на самом деле какие-нибудь числа.

$^{29}$ Здесь имеются в виду идеальные числа.

$^{30}$ Здесь имеются в виду идеальные числа.

$^{31}$ Как указывает Alex. (Comm. 82, 9 след.), главные из этих нелепостей сводятся к тому, что в таком случае идеальные числа будут только суммами разного количества единиц, и, таким образом, качественно различные вещи будут объясняться лишь количественно разнящимися между собою принципами.

$^{32}$ Ross (I, 200) рекомендует, следуя Bywater'у, читать вместо αί αύται—αύται, чтобы речь шла не о «тех же самых единицах», а о самих единицах, находящихся в данном числе.

$^{33}$ Ибо, согласно основному предположению, каждая единица будет индивидуально своеобразна (единицы — разнородны) и будет возникать из основного единого и основной неопределенности («неопределенная двойка», как принцип такой неопределенности).

$^{34}$ Опять — поскольку единицы разнородны, как они могут связаться в единое целое?

$^{35}$ Поскольку единицы — тем более числа — признаются качественно различными, нужно при объяснении мира исходить из них, — по образцу натурфилософов, принимавших несколько различных элементов (не возводя их к одной материи). Между тем, Платоновская школа считает первоначальное единое как бы однородным со всеми единицами, которые из него получаются, ибо каждая из этих единиц являет основную природу единого (Аристотель едва ли правильно представляет себе установленную Платоном роль единого: на самом деле единое у Платона — творческий принцип, создающий на основе неопределенности «диады» (двойки) конкретные числа). Отсюда созданные единым единицы — и точно так же идеальные числа — не могут быть самостоятельными сущностями. Очевидно поэтому, что если единое должно быть началом, то у него должно быть несколько значений (в одном случае единое — только общий род для качественно различных единиц, тогда причинами вещей будут эти качественно различные единицы; в другом — единое есть нечто существующее в себе, — тогда отдельные единицы не являются сущностями, и началом будет только это единое).

$^{36}$ Дается критика той дедукции, которую в Платоновской школе получают основные роды геометрических величин (как говорит Ross, они были бы идеями, если бы идеями не признавались в теперешней стадии развития школы только числа): величины, это — непосредственно следующие за идеями «идеальные» моменты, которые выводятся из «идей» — идеальных чисел 2, 3, 4, с одной стороны, и из видоизменений основного материального принципа (большого и малого) — с другой: линии — из длинного и короткого, плоскости — из широкого и узкого, тела — из глубокого и низкого.

$^{37}$ Аристотель опять говорит от имени Платоновской школы.

$^{38}$ μήκη означает величины одного измерения, иногда = линии.

$^{39}$ В противоположность «глубокому» мы бы сказали «мелкого».

$^{40}$ При наличии в основе каждого из этих геометрических родов (линии, плоскости, тела) особых начал.

$^{41}$ В основных геометрических определениях — линии, плоскости, теле.

$^{42}$ Имеются в виду источники геометрических определений: длинное и короткое и т.д.
$^{43}$ Ross в результате анализа этого места (I 203—208; ср. также I 189) предлагает другую пунктуацию, которая дает такой смысл: «... с этим родом бытия и боролся Платон, как с геометрическим учением, а началом линии называл (и неоднократно он это <начало> указывал) неделимые линии» ('αλλ' έκάλει αρχήν γραμμής — τώτο δέ πολλάκις έτίει — τάς άτόμους γραμμάς, α 20—22).

$^{44}$ От имени Платоновской школы.

$^{45}$ Имеется в виду, как явствует из непосредственно следующего, причина целевая. Ross не считает здесь нужным изменять чтение рукописей, указывая, что наша фраза хочет только подчеркнуть общее значение этой причины («to emphasize its importance») для научного познания. Может быть, даже, чтобы определеннее выявить основной смысл текста, достаточно более тесно слить две рядом стоящие фразы: «... что касается имеющей значение для наук (играющей роль для наук) причины, ради которой творит...» — Rolfes предлагает читать: «та причина, которая имеет значение для некоторых наук» (Zeller — для действенных наук), но Никомахова «Этика» начинается словами: «Всякое искусство и всякое исследование стремится, по-видимому, к какому-нибудь благу».

$^{46}$ Дословно: признать более математической <чем должно>.

$^{47}$ Лучше, может быть, переставить: и скорее образует отличительное свойство сущности и материи, сказываясь о них, нежели материю.

$^{48}$ Указываемые «физиологами» первые различия «вещественной» материи восходят к основному устанавливаемому Платоновскою школой отличию материи — «большому и малому», потому что в них также есть характерные для большого и малого моменты избытка и недостатка.

$^{49}$ Дословно: «если указанные свойства будут движением». Под «указанными свойствами» разумеется большое и малое. Jaeger, впрочем, предлагает вместо εσται ταύτα читать έστ' ένταα, и тогда получится смысл: «И что касается движения, если в нашем мире есть движение, то очевидно, что идеи будут двигаться».

$^{50}$ Т.е. если согласиться, что каждому устанавливаемому роду соответствует идея. Восходя ко все более общим родам, мы тогда получим, наконец, самый общий род — сущего, который, однако, при платоновской точке зрения, будет не единым всеобъемлющим бытием, а самостоятельной, отдельно существующей идеей.

$^{51}$ Формальное различие между всеобщим и родом — то, что род всегда является чем-то общим, но не — наоборот, так что понятие общего шире понятия рода; ср. Bonitz, Comm. 299—300 (к Ζ глава 3). Как раз в понятии сущего и понятии единого Аристотель видит не род, а всеобщее.

$^{52}$ А, может быть, у элементов? См. Bon. Comm. 125 («... quod 992 b 19 scribit μή διελόντας idem significat atque hoc loco — 993 а 8 — ταύτα»). По Ross'y, I, 189 (XX) речь идет об различных значениях бытия.

$^{53}$ Дословно: «Ведь ясно, что нельзя существовать до этого <познания> (т.е. до познания элементов всего сущего), зная что бы то ни было раньше <его>».

$^{54}$ Трудно уловить оттенок различия между προειδέναι («должно знать заранее») И είναι γνώριμα (дословно: «должны быть известны»). Комментаторы, начиная с Alex., указаний не дают. Не создается ли требуемый оттенок через конкретное значение γνώριμος — «знакомый», «близкий», отсюда — понятный?

$^{55}$ Сказано по адресу Платоновского учения о припоминании в «Меноне» и «Федоне».

$^{56}$ Bonitz в Comm. (стр. 125) формулирует: «и пусть мы допустим, что у нас (благодаря прирожденному знанию) имеются такие начальные положения; все же о них поднимется спор и сомнение, которое может быть разрешено только доводами, полученными из предшествующего познания».

$^{57}$ Дословно: «и ни один из тех, которые <нам> известны». — Ross отмечает, что филологи <еще сейчас, как во времена Аристотеля, спорят, какой звук (в разные эпохи) обозначался у греков знаком ζ.

$^{58}$ Воп. и Christ принимают предложенную Schwegler'ом поправку ταύτα «из одних и тех же», вместо ταύτα — «из этих» (т. е. платоновских). Область всего бытия объясняется из пригодных для всего элементов, подобно тому, как область звуков — из элементов, пригодных для этой области.






\newpage
\section{Книга 10}

\epigraph{
Никто из прежних философов не установил какого-либо начала, кроме указанных выше четырех родов причин, но и эти причины они скорее смутно предугадали, чем познали.
}{Парафраз А.В. Кубицкого}

\alignedmarginpar{$^1$ «Первая философия» здесь в смысле «основная философия» (т.е. наука о первых началах, метафизика), как Аристотель неоднократно употребляет этот термин (Met. Ε 1, 1026 а 16, Phys. А 9, 192 а 36 и др.).}
\alignedmarginpar{$^2$ Ввиду накопления однозначных выражений (дословно: «будучи вначале и впервые в молодых годах»), Ross рекомендует читать: «будучи в молодых годах и при начале» (άτε νέα τε χα! κατ'άργάς ούσα), а «и впервые» (και τό πρώτον) считает позднейшим пояснением (глоссой) для «при начале» (κα! κατ'αοχάς).}
\alignedmarginpar{$^3$ Аристотель говорит — через «логос», имея в виду указываемое Эмпедоклом соотношение отдельных материальных элементов в составе кости. При многочисленности значений термина «логос», его нельзя здесь передавать через субъективное «понятие», хотя, с другой стороны, «соотношение» (у Ross'a — ratio) плохо вяжется с моментом субстанциальности, который характерен для привлекаемой Аристотелем «сущности» и «сути бытия».}
\alignedmarginpar{$^4$ Дословно: «или уж — ни у какой». Здесь я читаю по тексту Christ'a, который ближе к чтению рукописей, так как не вижу решающих оснований для предложенной Воn. и принятой Ross'ом конъектуры, по которой вместо α σαρκός α των άλλων (εκάστου) είναι τον λόγον ή μηδενός, надлежит читать α σάρκας α των άλλων εκαστον είναι τον λόγον, ή μηδέ εν. Непонятно, почему при первом чтении требуется (Ross I, 213) мысленно добавлять ούσίαν или φύσιν, а нельзя понимать указанное место (как это и выражено в переводе) так: «указанное соотношение должно быть и у мяса и у каждой из других вещей» и т.д.}
\alignedmarginpar{$^5$ Под этим «прежним выяснением» надо иметь в виду не «все, что о высших началах вещей исследовали или смутно предугадывали древние философа», как думает Воn. (Comm. 127), но то, что раньше было уже сказано о первых началах самим Аристотелем (см. в первую очередь «Физику», книга II, главы 3, 7).}
\alignedmarginpar{$^6$ Alex. поясняет: «Затруднения и их решения в отношении начал становятся исходным пунктом для решения последующих затруднений» (Comm. 100, 3—4).}
Что все, по-видимому, стараются найти указанные в физике причины и помимо этих причин мы не могли бы указать ни одной, — это ясно уже из того, что было сказано раньше. Но ставят вопрос о них нечетко. И в известном смысле все они раньше указаны, а с другой стороны — отнюдь нет. Словно лепечущим языком говорит обо всем первая философия $^1$, будучи в молодых годах и при начале (своего существования). $^2$ Ведь и Эмпедокл говорит, что кость существует через соотношение $^3$, а это означает «суть бытия» и сущность вещи. Но подобным же образом такое «соотношение» (частей) должно быть и у мяса и у всякой другой вещи или уж — ни в одном случае (без исключения). $^4$ По этой причине следовательно будет существовать и мясо, и кость, и всякий другой предмет, а не через материю, которую он указывает, — через огонь, землю, воду и воздух. Но с этим он необходимо бы согласился, если бы (так) стал говорить кто другой, сам же он этого отчетливо но сказал. — Выяснение такого рода вопросов произведено и раньше. $^5$ А все, что по этим же вопросам может вызвать затруднения, мы пройдем снова. Этим путем мы может быть найдем какие-нибудь благоприятные указания для позднейших затруднений. $^6$

\end{document}
